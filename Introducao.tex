\chapter{Introdu��o}
\label{cap:introducao}

\section{Motiva��o}
\label{sec:motivacao}
Existem diversas finalidades para cria��o de agentes inteligentes para jogos. Podemos enumerar algumas como: adapta��o a diferentes tipos de jogador, gerar diferentes experi�ncias em cada nova partida, gerar grandes dificuldades entre outras.
\newline
Esse trabalho explora a cria��o de agentes inteligentes que compitam com jogadores humanos no sistema de batalhas \textit{Pok�mon} no ambiente \textit{Pok�mon Showdown}.
\newline
Um dos grandes desafios na cria��o desses agentes � a adaptabilidade. Os agentes ser�o treinados em batalhas com times rand�micas, al�m de precisar se adapatar ao seu pr�prio time, o agente n�o conhecer� o time advers�rio at� que o jogador advers�rio mostre um de seus \textit{Pok�mons} e use uma de suas habilidades.
\newline
Por causa da caracter�stica de desconhecer o time advers�rio, dificulta bastante a cria��o de grafos de poss�veis jogadas do advers�rio, pois o agente precisaria predizer os poss�veis \textit{Pok�mons} advers�rios e suas caracter�stica. Desse modo dificulta bastante t�cnicas baseadas em grafos de jogadas como a t�cnica utilizada no famoso \textit{Deep Blue} que segundo o trabalho \textit{Deep Blue System Overview} \cite{deepblue1} "O \textit{Deep Blue} � um massivo sistema paralelo para realiza��o de busca em �rvores de jogos de xadrez".

\section{Objetivos}
\label{sec:objetivos}

\section{Principais contribui��es}
\label{sec:principais}