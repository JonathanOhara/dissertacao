\chapter{Metodologia}
\label{cap:proposta}

Para atingir os objetivos propostos ser�o implementados tr�s agentes. Dois desses ser�o agentes que ir�o ser submetido a uma grande quantidade de treino para que possam chegar a patamares s�lidos. O terceiro ser� um agente baseado em grafos com diferentes profundidades onde pode ser visto a evolu��o no ranquemanto com o aumento de profundidade de �rvore de decis�o.

\begin{itemize}

	\item \textbf{Agente 1 com neuroevolu��o}. O primeiro agente persiste em um rede neural onde os pesos de sua rede ser�o ajustados por um algoritmo evolutivo.
	
	\item \textbf{Agente 2 com aprendizado por refor�o}. Esse agente ir� se ajustar atrav�s de estimulos positivos e negativos que regular�o sua decis�o dentro do jogo.
	
	\item \textbf{Agente 3 baseado em grafo de decis�o}. Utilizar� uma �rvore de decis�o onde os poss�veis movimentos ser�o mapeados. O algoritmo tentar� prever poss�veis inc�gnitas do advers�rio assumindo o pior cen�rio poss�vel para cada vari�vel n�o conhecida..
	
\end{itemize}

\section{Treino e aprendizado}

O primeiro agente que ser� testado contra jogadores humanos � o agente 3(grafo). Ser� criado v�rias vers�es desse agente, cada vers�o ter� uma quantidade diferente de profundidade em sua �rvore de escolhas. Cada vers�o do agente ser� submetido a uma s�rie da batalhas que s� ter� fim quando seu ranqueamento se estabilizar.

\section{Avalia��o de resultado}
%falar sobre quantificar a vitoria