\documentclass{ufabc}

\usepackage{amscd}
\usepackage{amsfonts}
\usepackage{amssymb}
\usepackage{epsf}
\usepackage{graphicx}
\usepackage{algorithm}
\usepackage{algpseudocode}
\usepackage{pifont}
\usepackage{indentfirst}
\usepackage{amsmath}
\usepackage[only,ninrm,elvrm,twlrm,fivrm,sixrm,sevrm,egtrm,egtit,tenrm,tenit,twlit,frtnrm]{rawfonts}
\usepackage{titlesec}

\instituicao{Universidade Federal do ABC\\Centro de Matem�tica, Computa��o e Cogni��o (CMCC)}
\curso{P�s-Gradua��o}{Ci�ncia da Computa��o}{Mestre}
\titulo{Agentes inteligentes para batalhas Pok�mon}
\autor{Jonathan Ohara}{de Araujo}
\orientador{Orientador: Prof. Dr. Fabr�cio Olivetti de Fran�a}
\coordenador{Prof. Dr. Jo�o Paulo Gois}	
\documento{Qualifica��o}
\convidados{Prof. Dr. Fabr�cio Olivetti de Fran�a}{Prof.Dr. Jo�o Paulo Gois}{Prof.Dr. Andr� Luiz Brand�o}{Prof.Dr. Denise Hideko Goya (Suplente)}
	
\date{03}{2017}

\setcounter{secnumdepth}{5}
\MakeRobust{\Call}

% novos comandos para qualificacao
\newcommand{\kCC}{\emph{kCC}}
\newcommand{\kCH} {\emph{kCH}}
\newcommand{\uCC}{\emph{1CC}}
\newcommand{\uCH} {\emph{1CH}}
\newcommand{\dCC}{\emph{2CC}}
\newcommand{\dCH} {\emph{2CH}}
\newcommand{\PTH}{\mathcal{P}}
\newcommand{\Tau}{\mathcal{T}}
\newcommand{\NP}{{\rm NP}}

\begin{document}
	\maketitle

	\pagenumbering{roman}
	
%	\begin{resumo}
Esse trabalho explora a cria��o de agentes inteligentes competitivos em um ambiente onde dezenas de milhares de jogadores humanos competem diariamente para melhorar sua coloca��o no sitema de ranqueamento do jogo Pok�mon Showdown.

Escolher o melhor algoritmo e a melhor forma de aprendizado para esse agente � um dos grandes desafios desse trabalho. Outro importante aspecto que o agente precisa se adaptar � a grande variedade de composi��es de times e Pok�mons que o agente e o seu advers�rio pode montar.

Para obter a melhor flexibilidade o agente ser� submetido somente a batalhas rand�micas onde as composi��es de equipes s�o todas aleat�rias e a avalia��o de performance do agente ser� feita atrav�s do sistema de ranqueamento do pr�prio jogo.
\end{resumo}
%	\begin{resumoingles}
This work explores the creation of an API (Application Programming Interface) to the online strategy game called Pok\'{e}mon Showdown!, allowing researchers to create autonomous agents to compete against human players in an extremely competitive environment with thousands of active players at all the time. The creation of this API can enable searches of new artificial intelligence techniques and comparision of different techniques when submitted to games with humans.

In this work we discuss and develop agents for this game using different Monte Carlo search tree techniques. There some important aspects need to be taken care during the creation of agents for this game such as: incompleteness and imperfection of information and simultaneity of choices are some of the main challenges.
\end{resumoingles}	

	\tableofcontents
	
	\listoffigures
	
	\listoftables

	\newpage
	
	\pagenumbering{arabic}

	\chapter{�rvore de jogos}
\label{chap:arvoreDeJogos}

\section{Introdu��o}

Nesse cap�tulo vamos discutir sobre tomada de decis�o em jogos e �rvore de decis�es. Em �rvores de decis�es n�s focaremos em 2 diferentes m�todos de buscas: t�cnicas baseadas em \textit{minimax} e t�cnicas baseadas em Monte Carlo.

O cap�tulo a seguir est� organizado do seguinte modo. Primeiro, a se��o \ref{sec:tomadaDecisao} ir� explicar o b�sico da id�ia de tomada de decis�o e os elementos envolvidos nela. Na se��o \ref{sec:arvoreDecisao} demonstra como decis�es podem ser modeladas em computa��o. Em \ref{sec:minimax} ser� discutido o conceito de \textit{minimax}, algumas t�cnicas e melhorias. Na se��o \ref{sec:MCTS} ir� expor sobre o m�todo de Monte Carlo e sua aplica��o em �rvores de busca. Por fim em \ref{sec:asdasd} ser� discutido alguns trabalhos relacionados a �rvore de busca de Monte-Carlo.

\section{Tomada de decis�o}
\label{sec:tomadaDecisao}

� muito comum associar intelig�ncia para jogos, mais especificadamente intelig�ncia artificial para jogos, com a habilidade em que os elementos do jogo (jogadores n�o humanos aliados ou inimigos, objetos, fases por exemplo) tem para tomar decis�es dados certas situa��es. Apesar disso, a tomada de decis�o n�o � o �nico componente de uma intelig�ncia artificial, na se��o \ref{sec:tomadaDecisao} vamos discutir mais profundamente sobre tomada de decis�o e de outros elementos que permeiam essa �rea.

Tomada de decis�o � o processo de escolher uma a��o entre diversas possibilidades. Segundo \cite{introductionDecisionMaking} "Tomada de decis�o � o estudo do processo de identificar e escolher alternativas baseadas nas prefer�ncias do tomador de decis�es. Tomar uma decis�o implica que existem diferentes escolhas para serem consideradas, e, em tal caso, n�o queremos apenas identificar quais dessas alternativas s�o vi�veis, mas escolher a decis�o que melhor se encaixa com as nossas metas, objetivos, desejos, valores, e assim por diante."

Ainda na teoria do processo de tomada de decis�o, \cite{guidebookDecisionMaking} diz que o primeiro passo na tomada de decis�o � estabelecer quem �(s�o) o(s) tomador(es) de decis�o e o(s) \textit{stakeholder(s)} (partes afetadas ou interessadas), de modo a mitigar um poss�vel desacordo sobre a defini��o do problema, metas e crit�rios. O processo pode ser definidos no seguintes passos:

\begin{itemize}
	
	\item Definir o problema;
	
	\item Determinar os requisitos que a solu��o deve apresentar;

	\item Estabelecer objetivos que a solu��o do problema deve realizar;
	
	\item Identificar alternativas que ir�o solucionar o problema;
	
	\item Desenvolver crit�rios de avalia��o com base nos objetivos;
	
	\item Selecionar uma ferramenta ou m�todo para de decis�o;
	
	\item Aplicar a ferramenta ou m�todo para selecioanar a alternativa preferida;
	
	\item Validar se a resposta resolveu o problema.
	
\end{itemize}

Nos jogos, segundo \cite{Millington:2009:AIG:1795711} a entrada da tomada de decis�o � o conhecimento que tal personagem tem e a sa�da � a a��o a ser realizada. O conhecimento pode ser dividio em interno e externo. Conhecimento externo � a informa��o que o personagem tem sobre o ambiente em sua volta: posi��o dos outros personagens, o leiaute da fase, se um interruptor foi ligado, a dire��o que um barulho veio, e assim por diante. Conhecimento interno � a informa��o sobre o estado interno do personagem ou pensamentos internos como: sua sa�de, objetivos, seu passado, e assim por diante. Ainda segundo \cite{Millington:2009:AIG:1795711} a representa��o do conhecimento est� intrinsecamente ligado com a maioria dos algoritmos de tomada de decis�o.

\subsection{�rvore de decis�o}
\label{sec:arvoreDecisao}

�rvores de decis�o � uma maneira muito clara e natural de representar uma cadeia de decis�es e suas consequ�ncias. Segundo \cite{Millington:2009:AIG:1795711} �rvores de decis�o s�o r�pidas, fac�is de implementar, e simples de entender. Elas s�o a t�cnica mais simples de tomada de decis�o. Outra vantagem � a quest�o de ser modular e muito f�cil de ser criada.

Um modo muito comum de representar �rvores de decis�o para estrat�gias simples � o fluxograma. Segundo \cite{Millington:2009:AIG:1795711} as decis�es �rvore de decis�es representadas por fluxogramas tem normalmente duas resposta (sim e n�o por exemplo). Na imagem \ref{fig:fluxogramaInimigos} � mostrado um diagrama simples de estrat�gia para um inimigo, nesse fluxograma o personagem fica patrulhando at� achar um inimigo, ao ver um inimigo o personagem se aproxima se necess�rio e o ataca.

\begin{figure}[!h]
\centering
\includegraphics[width=12cm]{figures/fluxogramainimigo.png}
\caption[Fluxograma estrat�gia inimigo]{Fluxograma Simples de estrat�gia para inimigos.}
\label{fig:fluxogramaInimigos}
\end{figure}

\subsection{M�quina de estados finitos}
\label{sec:fsm}

Outro modo de representar comportamentos de certos elementos � atrav�s de m�quinas de estados finitos (do ingl�s Finite State Machine - FSM). \cite{wright2012finite} define o modelo conceitual de FSM da seguinte forma:

\begin{itemize}
	
	\item O sistema (objeto que ser� modelado como uma FSM) deve ser descrev�vel como uma m�quina finita de estados.
	
	\item O sistema precisa ter um n�mero finito de entradas e/ou eventos que podem disparar transa��es entre estados.

	\item O comportamento do sistema dado um ponto no tempo ddo estado atual e da entrada ou evento que ocorrer� naquele tempo.
	
	\item Para cada poss�vel estado do sistema, o comportamento � definido para cada entrada ou evento poss�vel.
	
	\item O sistema tem um estado inicial particular.
	
\end{itemize}

O modo mais comum de representar o processo de decis�o de personagens em jogos digitais � atrav�s de m�quina de estados finitos (\cite{diller2004behavior}). Ainda segundo \cite{diller2004behavior} desenvolvedores de jogos comerciais est�o muito mais interessados em uma "ilus�o de intelig�ncia" (\cite{diller2004behavior}). Muitas vezes os personagens ir�o agir de limitados modos diferentes, por isso m�quinas de estados finitos funcionam muito bem, como acontece no jogo Halo da Bungie Software, onde os guerreiros da ra�a Convenant(inimitogs) ficam no estado "patrulha" at� perceber (por som ou vis�o) um jogador, em seguida, ele ir� alternar para o modo ataque(\cite{Millington:2009:AIG:1795711}).

Essa t�cnica � muito empregada em jogos comerciais. Segundo \cite{diller2004behavior} desenvolvedores de jogos comerciais pensam em uma intelig�ncia artificial que entre entretenha o p�blico. Em um cen�rio em que o objetivo � fazer a melhor intelig�ncia poss�vel, ou o objetivo � vencer os melhores humanos em um determinado desafio, tais t�cnicas n�o funcionam t�o bem. Em cen�rios competitivos pode-se enumerar alguns problemas dessas t�cnicas como a reatividade e a incapacidade de prever futuros movimentos ou analisar uma cadeia de movimentos em conjunto.

\subsection{�rvore de jogos}

Apesar da semelhan�a com �rvore de decis�o muitos autores como \cite{nijssen2013monte} e \cite{du1995minimax} trata como coisas diferentes mas, com recursos equivalentes. Uma �rvore de jogo � uma representa��o de um estado em um jogo sequencial. Os n�s representa as posi��es do jogo e as arestas representa os poss�veis movimentos de uma posi��o. O n� raiz representa o a posi��o inicial e os n�s folhas (ou terminais) representam o fim de jogo (\cite{nijssen2013monte}).

Os jogos podem ser classificado de acordo com algumas propriedades (\cite{browne12asurvey}):

\begin{itemize}
	\item \textbf{Soma-zero}: A soma das recompensas (positivas ou negativas) de todos os jogadores resulta em zero, ou seja, o que um jogador ganhou � exatamente o que o outro jogador perdeu.
	\item \textbf{Informa��o}: completo ou incompleto, perfeita ou imperfeita. Jogos com informa��o completa s�o aqueles que o estado atual do jogo � totalmente observ�vel por todos jogadores. Informa��o perfeita se refere se o jogador tem a perfeita informa��o de tudo que j� ocorreu.
	\item \textbf{Determinista}: Se existe um fator probabil�stico em certas a��es (tamb�m conhecida como complitude, por exemplo incerteza sobre certas recompensas).
	\item \textbf{Sequencial}: Se as a��es escolhidas s�o aplicadas sequencialmente ou simultaneamente.
	\item \textbf{Dicreto}: discreto ou cont�nuo.
\end{itemize}

Ainda segundo \cite{browne12asurvey} jogos com dois jogadores que s�o soma-zero, de informa��o perfeita, determin�ticos, discreto e sequenciais podem ser descritos como jogos combinat�rios. Essa defini��o inclui jogos como: Go, Xadres e Jogo da Velha, assim como muitos outros.

\section{Minimax}
\label{sec:minimax}

O algiritmo \textit{minimax} � comumente utilizado em jogos de dois jogadores e baseados em turnos. Jogos do mundo real normalmente envolvem uma estrutura de recompensas em que, apenas as recompensas obtidas em estados terminais (jogadas que definem quem � o vencedor) descrevem com precis�o o qu�o bem cada jogador est� se saindo (\cite{browne12asurvey}).

\subsection{Conceito e algoritmo}

O algoritmo de \textit{minimax} tenta minimizar a poss�vel perda m�xima, para isso ele analisa todas as futuros poss�veis movimentos de cada jogador . Os autores \cite{campbell1983comparison} descrevem o algoritmo de \textit{minimax} do seguinte modo: O algoritmo assume que existem dois jogadores chamados $Max$ e $Min$, e atribui um valor para cada n� dentro da �rvore do jogo (e um particular para n� raiz) do seguinte modo: N�s folhas ou terminais propagam o valor \textit{minimax} recursivamente para todos os outros n�s. Se um n� n�o terminal $p$ do jogador $Max$ for escolhida, ent�o o valor de $p$ � o m�ximo valor dos filhos de $p$. Similarmente, se for o turno do jogador $Min$ ser� escolhido o menor valor dos sucessores de $p$.

O algoritmo b�sico do minimax \textit{minimax} pode ser representado por:

\begin{algorithm}[H]
\caption{Algoritmo Minimax}\label{euclid}
\begin{algorithmic}[1]
\Procedure{Minimax}{$node, depth, max$}
\If {$depth = 0$ \textbf{OR} node is terminal} 
	\State \Return heuristic
\EndIf
\If {$max$}
	\State $bestValue\gets -\infty$
	\For{each child \Pisymbol{psy}{206} $node$ }
		\State $value\gets \Call{Minimax}{child, depth -1, FALSE}$
		\State $bestValue\gets \Call{max}{bestValue, value}$
	\EndFor
\Else
	\State $bestValue\gets +\infty$
	\For{each child \Pisymbol{psy}{206} $node$ }
		\State $value\gets \Call{Minimax}{child, depth -1, TRUE}$
		\State $bestValue\gets \Call{min}{bestValue, value}$
	\EndFor
\EndIf
\State \Return bestValue
\EndProcedure
\end{algorithmic}
\end{algorithm}

Onde:

\begin{itemize}
	\item \textbf{node} N� atual da �rvore. Na primeira chamada � passado o n� raiz da �rvore, ou seja, o estado atual do jogo.
	\item \textbf{depth} Profundidade do n�. Inicialmente � passada a altura da �rvore.
	\item \textbf{max} Booleano representando se a jogada analisada � do jogador \textbf{Max} ou \textbf{Min}.
\end{itemize}

Em seu trabalho, \cite{du1995minimax} propuseram uma rela��o entre �rvores \textit{minimax} e jogos que podem ser aplicados o \textit{minimax}. Essa rela��o pode ser vista na tabela \ref{tab:minimaxtreegames}.

\begin{table}[]
\centering
\caption{�rvore \textit{minimax} e jogos}
\label{tab:minimaxtreegames}
\begin{tabular}{|l|l|}
\hline
�rvore minimax                     & �rvores de jogos                                \\ \hline
�rvore                             & Todas poss�veis configura��es do tabuleiro      \\
N� da �rvore                       & Configura��o do tabuleiro                       \\
Arestas de um n� Max at� um n� Min & Movimento pr�prio (movimento do jogador Max)    \\
Arestas de um n� Min at� um n� Max & Movimento advers�rio (movimento do jogador Min) \\
Valor do n�                        & A "nota" para certa configura��o de tabuleiro   \\
N� folha (terminal)                & resultado do jogo (vit�ria, derrota ou empate)  \\
Caminho solu��o                    & Sequ�ncia de Movimentos que leva a melhor sa�da \\ \hline
\end{tabular}
\end{table}

Na figura \ref{fig:minimaxjogodavelha} � mostrado a aplica��o do algoritmo de \textit{minimax} para um cen�rio do jogo da velha. Na imagem o jogador Max � representado por \textbf{X} e o jogador Min por \textbf{O}. Como dito anteriormente, primeiramente a �rvore � explorada (partindo da raiz) at� que se chegue em n�s terminais (indicando fim de jogo), em situa��es vitoriosas foi atribu�do o valor de 10, em empates 0 e em derrota o valor -10. Em seguida, os pais desses n�s folhas s�o preenchidos com o m�nimo ou o m�ximo valor dos n�s filhos. No primeiro n� de Min (primeiro n� da segunda fileira), por exemplo, existem dois n�s filhos, um com valor 10 e outro com valor -10, por ser um n� Min � atribu�do para esse n� o menor valor (no caso -10). A linha em azul mostra a melhor jogada no momento para o n� raiz.

\begin{figure}[H]
\centering
\includegraphics[width=16cm]{figures/minimax.png}
\caption[Minimax]{\textit{Minimax} aplicado a um cen�rio de jogo da velha.}
\label{fig:minimaxjogodavelha}
\end{figure}

Segundo \cite{browne12asurvey} existem duas varia��es do \textit{minimax} muito comuns de serem encontradas:

\begin{itemize}
	\item \textbf{Expectimax} generaliza \textit{minimax} para jogos estoc�sticos em que as transi��es de estado para estado s�o probabil�stica. O valor de um n� � a soma dos valores dos n�s filhos ponderados por suas probabilidades (possibilidade de um estado ocorrer).
	\item \textbf{Miximax} � semelhante ao \textit{expectimax} de apenas um jogador e � usado principalmente em jogos com informa��es n�o precisas. Ele utiliza uma estrat�gia predefinida para tratar a decis�o do oponente como n�s probabil�sticos.
\end{itemize}

\subsection{Problemas}

Em jogos com grande quantidade de escolhas e/ou com um grande n�mero de turnos (altura de �rvore) esses algoritmos n�o se mostram muito efetivos, pois cada escolha tem que ser avaliada at� que chegue em n�s terminais. Para a maioria dos jogos geralmente n�o � vi�vel pesquisar toda a �rvore at� o n� terminal como especificado pelas regras do jogo (\cite{campbell1983comparison}).

Podemos enumerar dois grandes problemas ao utilizar o \textit{minimax} com uma �rvore muito extensa. O primeiro deles � fazer toda a simula��o de poss�veis jogadas at� que se chegue em n�s folhas (considerando que os caminhos sejam finitos). O segundo problema ocorre quando uma �rvore tem um grande n�mero de n�s e uma altura muito grande, nesse caso a fun��o de avalia��o do \textit{minimax} tem que percorrer muitos n�s at� que se chegue na raiz a partir dos n�s folhas.

\subsection{Varia��es}

Uma estrat�gia para reduzir a quantidade de n�s analisados � utilizar t�cnicas de poda de �rvores. A poda de �rvore mais conhecida na literatura � a chamada \textit{alpha-beta}. Segundo \cite{knuth1976analysis} essa � t�cnica � usada geralmente para aumentar a velocidade de busca sem perder informa��o. Nessa t�cnica s�o ignorados os n�s e suas sub�rvores de jogadas incapazes de ser melhor do que movimentos j� conhecidos. Durante a an�lise das jogadas s�o definidas duas vari�veis \textit{alpha} e \textit{beta}. \textit{Alpha} representa o valor m�ximo que o jogador Max pode fazer e \textit{beta} a pontua��o m�nima do jogador Min. A cada avalia��o de n� esses limites s�o verificados, caso o valor da avalia��o n�o estiver entre esses limites o n� e todas suas �rvores s�o cortados. Que pode ser dado pelo seguinte algoritmo (na primeira chamada alpha vale $-\infty$ e beta $+\infty$):

\begin{algorithm}[H]
\caption{Algoritmo alphabeta}\label{euclid}
\begin{algorithmic}[1]
\Procedure{alphabeta}{$node, depth, currentPlayer, alpha, beta$}
\If {$depth = 0$ \textbf{OR} node is terminal} 
	\State \Return heuristic
\EndIf
\If {$max$}
	
	\State $bestValue\gets -\infty$
	\For{each child \Pisymbol{psy}{206} $node$ }
		\State $bestValue\gets $\Call{max}{bestValue, \Call{alphabeta}{child, depth -1, alpha, beta, FALSE}}
		\State $alpha\gets \Call{max}{alpha, value}$
		\If {$beta \leqslant alpha$} 
			\State \textbf{break}
		\EndIf
	\EndFor
	
\Else

	\State $bestValue\gets +\infty$
	\For{each child \Pisymbol{psy}{206} $node$ }		
		\State $bestValue\gets $\Call{min}{bestValue, \Call{alphabeta}{child, depth -1, alpha, beta, TRUE}}
		\State $beta\gets \Call{min}{beta, bestValue}$
		\If {$beta \leqslant alpha$} 
			\State \textbf{break}
		\EndIf
	\EndFor
	
\EndIf
\State \Return bestValue
\EndProcedure
\end{algorithmic}
\end{algorithm}

Para resolver o problema de simular um grande n�mero de jogadas e ao mesmo tempo diminuir a quantidade de n�s a serem analisados � poss�vel criar avalia��es de heur�sticas para n�s n�o terminais e assim delimitar a simula��o da �rvore at� uma determinada altura.

\subsection{Avalia��o de heur�stica}

Em jogos com pequenos n�meros de n�s na �rvore, a avalia��o de heur�stica n�o � t�o importante, um exemplo desse tipo de jogo � o jogo da velha, devido ao reduzido n�mero de n�s e a pequena altura da �rvore � mais seguro explorar a �rvore at� o final do que criar avalia��es de heur�sticas para n�s n�o terminais.

Segundo \cite{Nielsen:1994:HE:189200.189209} avalia��o heur�stica envolve ter um pequeno conjunto de avaliadores examinando a interface e avaliando a sua conformidade em rela��o ao resultado. Os autores \cite{DBLP:conf/aaai/ChristensenK86} discorrem que, em jogos de tabuleiro de duas pessoas a fun��o de heur�stica � vagamente caracterizado pela "for�a" do posicionamento de um jogador contra o outro. Ainda segundo \cite{DBLP:conf/aaai/ChristensenK86} pode-se dizer que uma fun��o de avalia��o de heur�stica tem duas propriedades:

\begin{itemize}
	\item Quando aplicado a um estado final (no caso de uma �rvore, um n� folha), a avalia��o de heur�stica tem que devolver o estado corretamente;
	\item O valor da fun��o de heur�stica � invari�vel ao longo de um caminho de solu��o �tima.
\end{itemize}

\subsection{Sistema Deep Blue}

Um dos sistemas mais famosos que utilizou as t�cnicas citadas foi o Deep Blue da IBM. O Deep Blue � um sitema feito para jogar xadrez com t�cnicas baseadas em \textit{minimax}. Al�m de utilizar o buscador alphabeta, o sistema conta com uma complexa fun��o de avalia��o de n�s terminais e analisava jogadas com 12 n�veis de profundidade \cite{755469}. O Deep Blue era composto por um \textit{chip} chamado \textit{Chess Chip} que � dividido em tr�s partes\cite{deepblue1}:

\begin{itemize}

	\item \textbf{Buscador alphabeta}. Menor parte do \textit{chip}, contendo apenas 5\% de seu total, esse componente � respons�vel por fazer a busca na �rvore de jogadas. O algoritmo utilizado � uma variante do \textit{alphabeta} chamado de \textit{minimum-window alpha beta search}, essa t�cnica foi escolhida por n�o ser necess�rio uma pilha de valores.
	
	\item \textbf{Gerador de movimentos}. Esse componente cont�m o \textit{array} bidimensional 8x8 referente a cada posi��o do tabuleiro do xadrez, ele tamb�m � respons�vel por realizar e verificar a legalidade dos movimentos de acordo com as regras do xadrez.
	
	\item \textbf{Fun��o de avalia��o}. Ocupando cerca de dois ter�os do \textit{chip} esse componente � respons�vel pela avalia��o dos movimentos. Cada poss�vel posi��o de cada pe�a � avaliado, essa avalia��o � baseada em quatro crit�rios: material, posi��o, seguran�a do rei e tempo \cite{ibmdeepblue}. Material � o quanto cada pe�a "vale". Por exemplo, um pe�o vale 1 enquanto uma torre vale 5. Posi��o quantifica o qu�o bom est�o posicionadas as pe�as, nesse c�lculo uma s�rie de aspectos � levada em conta, um deles � o n�mero de movimentos seguros que as pe�as podem fazer para atacar. Seguran�a do rei � calculo que leva em conta a prote��o do rei. Por final tempo, al�m de controlar o tempo da pr�pria jogada, fazer uma jogada r�pida d� ao advers�rio menos tempo para pensar em poss�veis jogadas.
	
\end{itemize}

\section{�rvore de Busca de Monte Carlo}
\label{sec:MCTS}

Nessa se��o vamos discutir sobre o m�todo de Monte Carlo e sua aplica��o, sobre �rvore de busca de Monte Carlo (do ingl�s Monte Carlo Tree Search - MCTS), uma heur�stica de bucas utilizada para estimar �rvores de decis�o, principalmente em jogos baseados em turnos e por fim veremos algumas melhorias e trabalhos relacionados na �rea. 

\subsection{M�todo de Monte Carlo}

O m�todo de Monte Carlo, do qual o algoritmo MCTS foi derivado, compreende o ramo da matem�tica experimental que concerne com experimentos utilizando n�meros aleat�rios (\cite{hammersley2013monte}). Segundo \cite{kleij2010monte} esse m�todo matem�tico se refere a utilizar amostras aleat�rias com o objetivo de estimar solu��es de problemas infact�veis de serem resolvidos analitcamente. 

O real uso dos metodos de Monte Calor como ferramenta de pesquisa resulta de trabalhos em bombas at�micas durante a segunda guerra, esse trabalho envolveu uma simula��o direta de problemas probabil�sticos relacionados com a difus�o aleat�ria de neutrons em material f�ssil (\cite{hammersley2013monte}). Ainda segundo \cite{hammersley2013monte} a possibildiade de aplicar m�todos de Monte Carlo para problemas determin�sticos foi notado por Fermi, von Neuman, e Ulam e popularizado imedaimente ap�s a guerra.

O m�todo de Monte Carlo possui diversas variantes, mas todas elas seguem esses passos: Dado uma probabilida de um evento acontecer chamado de $p$, uma quantiadde de n�meros ou conjuntos n�mericos alet�rios s�o gerados, a quantidade de vezes que esse conjunto aleat�rio disparou o evento divivido pelo quantidade de n�mero ou conjunto n�mericos gerados deve ser pr�ximo de $p$.

\subsubsection{Utilizando m�todo de Monte Carlo para calcular $\pi$}

Um exemplo bem difundido para utiliza��o do m�todo de Monte Carlo � para definir o valor de $\pi$. Calculamos a probabilidade aleatoria de pegarmos posi��es dentro do quadrado da figura \ref{fig:quadradoCirculoPi} e essa posi��o estar tamb�m dentro do c�rculo.

\begin{figure}[!h]
\centering
\includegraphics[width=12cm]{figures/piexample01.png}
\caption[Fluxograma estrat�gia M�todo de Monte Carlo aplicado para descobrir o valor de $\pi$ - 1]{Quadrado com lado de 2r contendo um c�rculo medindo r.}
\label{fig:quadradoCirculoPi}
\end{figure}

A �rea de um quadrao � dado pela seguinte equa��o:

\begin{equation}
A = l^2
\end{equation}

Onde $l$ � o lado do quadrado.

J� a �rea do c�rculo � dada pela seguinte equa��o:

\begin{equation}
A = \pi \times r^2
\end{equation}

Onde $r$ � o raio do c�rculo.

Substituindo os valores da equa��o temos que �rea do quadrado � $4r^2$ e a �rea do c�rculo � $\pi r^2$. Logo, a chance de aleatoriamente ser escolhida uma posi��o dentro do quadrado e essa posi��o tamb�m estar dentro do c�rculo � dado por:

\begin{equation}
p = \frac{\pi r^2}{4r^2}
\end{equation}

Isolando $\pi$ (valor que queremos descobrir) temos a seguinte equa��o:

\begin{equation}
\pi = 4p
\end{equation}

Com o aux�lio de um algoritmo � gerado 100.000 posi��es aleat�rias. Dessa amostra 78239 ficavam dentro do c�rculo. A imagem \ref{fig:quadradoCirculoPiAleatoria} faz uma demonstra��o gr�fica desse conjunto gerado, os pontos mais escuros mostram posi��es geradas fora do c�rculo, e os pontos mais claros os pontos dentro da �rea do c�rculo.

\begin{figure}[!h]
\centering
\includegraphics[width=12cm]{figures/montecarlopi.png}
\caption[M�todo de Monte Carlo aplicado para descobrir o valor de $\pi$ - 2]{Demonstra��o gr�fica de gera��o de 100.000 posi��es aleat�rias.}
\label{fig:quadradoCirculoPiAleatoria}
\end{figure}

Voltando a defni��o temos o seguinte: a quantidade de vezes que esse conjunto aleat�rio disparou o evento (78466) divivido pelo quantidade de n�mero ou conjunto n�mericos gerados (100.000) deve ser pr�ximo de $p$ ($\pi = 4p$ ou $p = \frac{\pi}{4}$. Substituindo o que descobrimos temos o seguinte:

\begin{equation}
\frac{\pi}{4} = \frac{78621}{100000}
\end{equation}

Isolando $\pi$ e resolvendo a equa��o temos o seguinte:

\begin{equation}
\pi = \frac{314484}{100000}
\end{equation}

Finalmente temos o valor aproximado de $\pi = 3,14484$.

\subsubsection{Utilizando m�todo de Monte Carlo para probabilidade de uma roleta}

Com o m�todo de Monte Carlo � bem f�cil calcula probabilidade mais complexas. A imagem \ref{fig:montecarloroleta} representa uma roleta que pode ser dividida em 4 partes: em duas dessas partes o valor ganho � de $+1$, em uma dessas quatro partes o valor ganho � $+2$ e na �ltima parte o valor � de $-2$. Nesse caso vamos usar o m�todo de Monte Carlo para descobrir qual a probabilidade de ap�s girar 20 vezes a roleta a soma dos pontos conseguidos ser menor que $0$.

\begin{figure}[!h]
\centering
\includegraphics[width=10cm]{figures/roleta.png}
\caption[M�todo de Monte Carlo aplicado a probabilidade de uma roleta]{Distrubui��o de pontos de uma roleta.}
\label{fig:montecarloroleta}
\end{figure}

Ser� feito 1.000.000 de simula��es, em cada simula��o ser� girado a roleta 20 vezes e os pontos ser�o somado. Ap�s a simula��es temos o seguinte que em 64020 das simula��es o resultado da soma resultou em valores negativos. Voltando a defini��o temos:

\begin{equation}
p = \frac{64020}{100000}
\end{equation}

Ou seja, a chance de ap�s rodar 20 vezes a roleta da figura \ref{fig:montecarloroleta} a chance da soma de pontos ser negativa � de $0,06402$.

\subsection{Avalia��o de n�s utilizando Monte Carlo}

O primeiro contato do m�todo de Monte Carlo com �rvore busca foi na avalia��o de n�s (\cite{nijssen2013monte}, \cite{Winands2008}). Essa avalia��o era chamada de avalia��o de Monte Carlo (do ingl�s Monte Carlo Evaluation - MCE).

Em vez de usar recursos dependentes do dom�nio, tais como a mobilidade ou vantagem de um material, uma s�rie de \textit{playouts} (chamado tamb�m de amostras, lan�amentos, reprodu��o ou simula��es) � iniciado a partir da posi��o a ser avaliada. O \textit{playout} � uma sequ�ncia de movimentos (semi-)aleat�rios. O \textit{playout} termina quando o jogo est� terminado e um vencedor pode ser determinado de acordo com as regras do jogo. A avalia��o de uma posi��o � determinada por combina��o dos resultados de todos os \textit{playouts} usando uma fun��o de agrega��o estat�stica. Isso pode ser, por exemplo, a pontua��o m�dia jogo (\cite{nijssen2013monte}).

Por n�o exigir muito conhecimento do dom�nio ela funciona muito bem em jogos onde n�o existe uma fun��o de avalia��o de n�s intermedi�rios muito bom (como em Go por exemplo), como no trabalho de \cite{brugmann1993monte} onde o MCE foi aplicado a um tabuleiro de 9x9. A grande desvantagem dessa t�cnica � a quantidade de tempo que ela leva para avaliar um determinado n� at� chegar em um n� folha, fazendo ela ser eficaz apenas em �rvores com profundidade de n�s baixa. Uma deriva��o direta dessa abordagem � a �rvore de busca de Monte Carlo.

\subsection{Monte Carlo em �rvores de busca}

Por d�cadas a busca alpha-beta tem sido a abordagem padr�o usado em programas de jogos soma-zero de dois jogadores como xadrez e damas (entre outros). Ao passar dos anos muitas melhorias foram propostas para essa solu��o. Por�m, em alguns jogos onde � dif�cil construir uma fun��o de avalia��o precisa (e.g., Go) a abordagem alpha-beta foi mal sucedida (\cite{Winands2008}).

O Algoritmo de busca de Monte Carlo ganhou muita notoriedade por apresentar uma alta competitividade ao enfrentar os melhores jogadores do jogo Go. Go � um jogo muito dif�cil para computadores jogar: tem alto fator de ramifica��o, a uma �rvore profunda, e n�o tem nenhuma fun��o de valida��o de herur�stica confi�vel para n�s n�o terminais (\cite{browne12asurvey}). Ainda segundo os autores nos �ltimos anos, MCTS tem alcan�ado grande sucesso em muitos jogos espec�ficos, jogos generalistas, complexos planejamentos do mundo real, otimiza��o e controle de probelmas, e se tornou uma importante ferramenta do pesquisador de intelig�ncia artificial.

Outro grande ponto do algoritmo de busca de Monte Carlo � o resultado positivo que ele obteve em jogos com informa��o incompleta e/ou imperfeita como nos trabalhos de \cite{Cai:2005:MCA:1066384.1066386} e \cite{misra2013markov}.

O m�todo de �rvore de busca de Monte Carlo se refere a encontrar solu��es �timas em um determinado dom�nio utilizando de amostras aleat�rias no espa�o de decis�o e construindo uma �rvore de busca de acordo com os resultados (\cite{browne12asurvey}).

A �rvore de busca de Monte Carlo � um algoritmo baseado em simula��o, frequentemente usado em jogos. A ideia principal � iterativamente rodar simula��es do n� raiz da �rvore at� um n� terminal, incrementalmente crescendo uma �rvore onde o n� raiz � o estado atual do jogo (\cite{tak2014monte}).

O algoritmo de MCTS � um processo iterativo que ocorre at� que um determinado crit�rio de parada seja alcan�ado. Segundo \cite{browne12asurvey}, geralmente essa itera��o � limitada por algum recurso computacional como tempo, mem�ria ou algum contador de itera��o. 

Basicamente o MCTS � uma �rvore que cresce de modo incremental e ass�ncrono como mostra a figura \ref{fig:mctsBasic}. O que define a expans�o das �rvores s�o as seguintes pol�ticas:

\begin{itemize}
	\item \textbf{Pol�tica de �rvore}: � a estrat�gia de qual n� escolher, segundo \cite{browne12asurvey} essa pol�tica tem que balancear escolher n�s pouco visitados e aprofundar n�s que parecem promissores. Um n� � expans�vel quando ele n�o � n� folha e ele ainda n�o foi visitado.
	
	\item \textbf{Pol�tica padr�o}:  � a estrat�gia de como rodar as simula��es at� sua conclus�o. Um caso bem simples de pol�tica padr�o � fazer movimento aleat�rios.
\end{itemize}


\begin{figure}[!h]
\centering
\includegraphics[width=14cm]{figures/mctsscheme.png}
\caption[Representa��o B�sica do MCTS]{Representa��o B�sica do MCTS \cite{5672398}.}
\label{fig:mctsBasic}
\end{figure}

Pode-se descrever cada itera��o do algoritmo em quatro passos b�sicos:

\begin{itemize}
	
	\item \textbf{Sele��o} Come�ando pelo n� raiz da �rvore, um n� filho � escolhido de acordo com a \textbf{pol�tica de �rvore} at� que seja encontrado um \textbf{n� expans�vel}. 
	
	\item \textbf{Expans�o} � adicionado um ou mais n�s filhos no n� escolhido de acordo com as poss�veis a��es naquele estado.

	\item \textbf{Simula��o} Simula o jogo a partir do(s) novo(s) n�(s) gerado(s) na fase de expans�o at� um o fim de jogo. As escolhas das a��es a partir desse novo n� � definido pela \textbf{pol�tica padr�o}. Conhecimentos do jogo pode ser adicionado para a escolha dessas a��es fazendo a simula��o mais real�stica (\cite{nijssen2013monte}).
	
	\item \textbf{Retropropaga��o} O resultado (valor) da simula��o � propagado para todos os pais do n� folha (at� o n� raiz) atualizando suas estat�sticas. Normalmente os valores s�o: vit�ria 1, empate $\frac{1}{2}$ e derrota 0.
	
\end{itemize}

A imagem \ref{fig:mctsScheme} mostra o processo do MCTS. Na imagem $V$ � o valor da recompensa (vit�ria, empate ou derrota) de um n� folha e $\Delta$ � o resultado da estat�sticas do n� (combina��o de recompensas positivas e negativas).

\begin{figure}[!h]
\centering
\includegraphics[width=\columnwidth]{figures/mctsprocess.pdf}
\caption[MCTS esquema]{MCTS esquema.}
\label{fig:mctsScheme}
\end{figure}

O algoritmo a seguir mostra o procedimento b�sico do MCTS:

\begin{algorithm}[H]
\caption{Algoritmo MCTS}
\label{alg:mcts}
\begin{algorithmic}[1]

\Procedure{Selecionar}{$no$}
	\While{ $no.filhos > 0$ }
		\State $no\gets$ \Call{PoliticaArvore}{$no$}
	\EndWhile	
	
	\State \Return $no$
\EndProcedure
\State
\Procedure{Simular}{$no$}
	\State $acao\gets no.acao$
	\While{ \textbf{not} $acao.isFimDeJogo$}
		\State \Call{Executar}{$acao$}
		\State $acoes\gets$ \Call{LerAcoes}{ }
		\State $acao\gets$ \Call{EscolherAcaoAleatoria}{acoes}
	\EndWhile
	
	\State $valor\gets$ $\frac{1}{2}$
	\If{ $acao.isVitoria$ }
		\State $valor\gets$ $1$
	\ElsIf{ $acao.isDerrota$ }
		\State $valor\gets$ $0$
	\EndIf

	\State \Return $valor$
\EndProcedure
\State
\Procedure{MCTS}{$noAtual$}
	\While{recurso computacional}
		\State $noSelecao\gets$ \Call{Selecionar}{$noAtual$}
		\State $noExpansao\gets$ \Call{Expandir}{$noSelecao$}
		\State $valorResultado\gets$ \Call{Simular}{$noExpansao$}
		\State \Call{RetroPropagar}{$noExpansao$, $valorResultado$}
	\EndWhile
	
	\State \Return \Call{MelhorFilho}{noAtual}
\EndProcedure
\end{algorithmic}
\end{algorithm}

Onde:

\begin{itemize}
	
	\item $noAtual$: N� que representa o estado atual do jogo. No primeiro turno � passado o n� raiz da �rvore.
	
	\item \Call{MelhorFilho}{no}: Escolhe o melhor n� filho. O conceito de melhor filho varia entre as diferentes implementa��es do MCTS.
	
\end{itemize}

Segundo \cite{schadd2009monte} ao finalizar o MCTS existe quatro crit�rios de como escolher o melhor filho:

\begin{itemize}
	
	\item \textbf{Filho m�ximo}: Seleciona o filho com valor mais alto.
	
	\item \textbf{Filho robusto}: Seleciona o filho com o maior n�mero de visitas.
	
	\item \textbf{Filho m�ximo-robusto}: Selecione o filho que tem tanto o maior n�mero de visita e o valor mais alto. Se n�o existir esse filho, � melhor continuar a busca at� achar um filho m�ximo-robusto do que escolher um filho com baixo n�mero de visitas.
	
	\item \textbf{Filho seguro}: Selecione o filho com menor chance de ter resultado negativo.
	
\end{itemize}

\subsection{Upper Confidence Bound for Trees (UCT)}

A implementa��o mais comum do algoritmo MCTS � o \textit{Upper Confidence Bound for Trees} (UCT) (\cite{browne12asurvey}). 

Um dos grandes dilemas na defini��o de pol�tica de �rvore � como balancear as escolhas entre explorar novos n�s para fugir de m�ximos locais e aprofundar de uma sub�rvore existente podendo achar melhores resultados ou garantindo uma maior robustez na avalia��o de um n�. Segundo \cite{auer2002finite} o Upper Confidence Bound for Trees (UCT) resolve essa quest�o de maneira �tima at� um fator constante.

O algoritmo UCT pode ser decomposto em duas partes: MCTS e Upper Confidence Bounds (UCB). Sua primeira implementa��o formal foi em 2006 por Kocsis e Szepesvari \cite{kocsis2006bandit}. O algoritmo UCB entra como uma pol�tica de �rvore para implementa��o de MCTS.


\subsubsection{Upper Confidence Bounds (UCB)}

O algoritmo mais tradicional de Upper Confidence Bounds � o UCB1 (\cite{auer2002finite}) que foi inicialmente aplicado para o problema \textit{multiarmed bandit}. Segundo \cite{auer2002using} o termo \textit{multiarmed bandit problem} ou problema dos ca�a-n�queis (ou mais precisamente "problema dos K ca�a-n�queis") reflete o problema de um apostador em uma sala com v�rias m�quinas de ca�a n�queis. Em cada tentativa o apostador tem que decidir em qual m�quina ele quer jogar. Para maximizar o ganho total ou recompensa, sua escolha se baseia em recompensas anteriormente coletadas em cada m�quina.

O UCB1 pode ser definido pela seguitne equacao:

\begin{equation}
UCB1 = \overline{x}_{j} + \sqrt{\frac{2 \ln n}{n_{j}}}
\end{equation}

Onde: $\overline{x}_{j}$ � a m�dia da recompensa paga pela m�quina $j$, $n_{j}$ � o numero de vezes em que foi jogado na m�quina $j$ e $n$ � a soma de quantas vezes foram em jogadas em todas as m�quinas.

\begin{itemize}
	
	\item: $j$: a escolha que est� sendo analisada. No exemplo a m�quina ca�a-n�quel.
	
	\item: $\overline{x}_{j}$: a m�dia da recompensa paga pela m�quina $j$.
	
	\item $n_{j}$: a quantidade de vezes que foi jogado na m�quian $j$.
	
	\item $n$: A soma de quantas vezes foi jogado em cada m�quina.
	
\end{itemize}

Nessa equa��o fica claro como est� distribu�do o fator explora��o ($\sqrt{\frac{2 \ln n}{n_{j}}}$) e o fato aprofundamento ($\overline{x}_{j}$). A imagem \ref{fig:exploreandexploitinequation} ilustra essa separa��o.

\begin{figure}[!h]
\centering
\includegraphics[width=12cm]{figures/exploreandexploit.png}
\caption[UCB1 explora��o e aprofundamento na equa��o]{UCB1 explora��o e aprofundamento na equa��o.}
\label{fig:exploreandexploitinequation}
\end{figure}

Por esse motivo � comum encontrar na literatura a equa��o UCB1 da seguinte forma:

\begin{equation}
UCB1 = \overline{x}_{j} + C \sqrt{\frac{2 \ln n}{n_{j}}}
\end{equation}

Onde $C$ � uma constante que regula o valor da explora��o, ou seja, caso queira aumentar a explora��o basta $C > 1$, caso queira diminuir a explora��o $1 > C > 0$.

\subsubsection{Pol�tica de �rvore UCB1}

A aplica��o do UCB1 como pol�tica de �rvore funciona do seguinte modo: no processo de sele��o a escolha pode ser modelada com um problema \textit{multiarmed bandit} independente. A utilizica��o do MCTS com qualquer pol�tica de �rvore UCB � chamada de UCT.

Uma varia��o proposta por \cite{kocsis2006improved} chamada de \textit{plain UCT} provou ter resultados muito superiores ao UCT comum. Ainda segundo \cite{kocsis2006improved} com essa modifica��o a chance de selecionar o melhor movimento  converge em 1 e, atrav�s de testes emp�ricos foi observado que � mais r�pido que outros algoritmos de busca em �rvore como o \textit{alpha-beta}.

A equa��o do \textit{plain UCT} � definida da seguinte forma:

\begin{equation}
UCT = \overline{x}_{j} + 2C_{p} \sqrt{\frac{2 \ln n}{n_{j}}}
\end{equation}

Onde $C_{p} > 0$ � uma constante. O valor � sugerido como $C_{p} = \frac{1}{\sqrt{2}}$ e pode ser ajustado para mais ou menos para regular o n�vel de explora��o.

\subsubsection{Algoritmo UCT}

Baseado no algoritmo MCTS previamentes explicado, vamos fazer algumas altera��es para implementgar o UCT. Primeiro o c�digo do m�todo principal:

\begin{algorithm}[H]
\caption{Algoritmo UCT}
\label{alg:mcts}
\begin{algorithmic}[1]
\Procedure{UCT}{$noAtual$}
	\While{recurso computacional}
		\State $noSelecao\gets$ \Call{Selecionar}{$noAtual$}
		\State $noExpansao\gets$ \Call{Expandir}{$noSelecao$}
		\State $valorResultado\gets$ \Call{Simular}{$noExpansao$}
		\State \Call{RetroPropagar}{$noExpansao$, $valorResultado$}
	\EndWhile
	
	\State \Return \Call{UCB1}{noAtual}
\EndProcedure
\end{algorithmic}
\end{algorithm}

M�todo Selecionar e UCB1:

\begin{algorithm}[H]
\caption{Algoritmo UCT - Selecionar e UCB1}
\label{alg:mcts}
\begin{algorithmic}[1]
\Procedure{UCB1}{$no$}
	\State $melhorValor\gets -\infty$ 
	\State $melhorFilho\gets$ \textbf{null}
	\For{each child \Pisymbol{psy}{206} $no$ }
		\State $valor\gets child.value + C \sqrt{\frac{2 \ln no.visitas}{child.visitas}}$
		\If{ $valor > melhorValor$ }
			\State $melhorValor\gets valor$
			\State $melhorFilho\gets child$
		\EndIf
	\EndFor
	
	\State \Return $melhorFilho$
\EndProcedure
\State
\Procedure{Selecionar}{$no$}
	\While{ $no.filhos > 0$ }
		\State $no\gets$ \Call{UCB1}{$no$}
	\EndWhile
	
	\State \Return $no$
\EndProcedure
\end{algorithmic}
\end{algorithm}

E por fim o retropropagar:

\begin{algorithm}[H]
\caption{Algoritmo UCT - Retropropagar}
\label{alg:mcts}
\begin{algorithmic}[1]
\Procedure{Retropropagar}{$no$, $valor$}
	\State $no.visitas\gets no.visitas + 1$ 
	\State $no.valor\gets no.valor + valor$

	\If{ no.pai \textbf{not} null }
		\State \Call{Retropropagar}{$no.pai$, $valor$}
	\EndIf

\EndProcedure
\State
\end{algorithmic}
\end{algorithm}

\subsubsection{Exemplo UCT}

Dado o algoritmo explicado na subse��o anterior, vamos executar um exemplo. Partindo da seguinte �rvore da figura \ref{fig:mctsexample01} ser� executado 1 itera��o passo a passo. Na imagem os valores dentro do n� representam valor (soma das recompensas) / visitas (quantas vezes o n� foi visitado).

\begin{figure}[H]
\centering
\includegraphics[width=10cm]{figures/mctsexample01.pdf}
\caption[MCTS Exemplo - �rvore inicial]{MCTS Exemplo - �rvore inicial.}
\label{fig:mctsexample01}
\end{figure}

Come�ando pelo n� raiz chamamos a fun��o UCT (ser� utilizado $C = \sqrt{2}$). Aplicando o m�todo UCB1 a todos filhos do n� raiz a �rvore fica do seguinte modo \ref{fig:mctsexample02} (valores de UCB1 est�o a esquerda da �rvore).

\begin{figure}[H]
\centering
\includegraphics[width=10cm]{figures/mctsexample02.pdf}
\caption[MCTS Exemplo - UCB1 aplicado]{MCTS Exemplo - UCB1 aplicado.}
\label{fig:mctsexample02}
\end{figure}

O n� com maior valor de UCB1 n�o tem filhos, ou seja, ele � expans�vel. Expandido ele e fazendo a simula��o teremos a seguinte �rvore como indica a figura \ref{fig:mctsexample03} (a simula��o resultou em vit�ria).

\begin{figure}[H]
\centering
\includegraphics[width=10cm]{figures/mctsexample03.pdf}
\caption[MCTS Exemplo - Expans�o e simula��o]{MCTS Exemplo - Expans�o e simula��o.}
\label{fig:mctsexample03}
\end{figure}

O �ltimo passo da nossa itera��o � a retropopaga��o. A imagem \ref{fig:mctsexample04} mostra como ficou a �rvore depois da �ltima adi��o e simula��o.

\begin{figure}[H]
\centering
\includegraphics[width=10cm]{figures/mctsexample04.pdf}
\caption[MCTS Exemplo - Expans�o e simula��o]{MCTS Exemplo - Expans�o e simula��o.}
\label{fig:mctsexample04}
\end{figure}

Ap�s isso voltamos para o in�cio do la�o e caso ainda tenha recurso computacional e come�amos novamente utilzando a nova �rvore.

\subsection{Melhorias MCTS}

Um aprimoramento para o UCT para quando se tem muitas a��es parecidas � a chamada grupos de movimentos. Proposto por \cite{childs2008transpositions} essa melhoria diminui consideravelmente a quantidade de ramos da �rvore Monte Carlo. A t�cnica consiste em criar uma nova camada onde todas as poss�veis a��es s�o separadas em grupos e o UCB1 � usado para selecionar qual grupo ser� escolhido.

	\chapter{Aprendizado de jogos com humanos}
\label{chap:aprendizadoJogosComHumanos}

\section{Introdu��o}

Nesse cap�tulo vamos discutir sobre aprendizagem de m�quina, modos de aprendizado e estudos que utilizam humanos de alguma forma durante o treinamento das redes neurais. O cap�tulo est� organizado da seguinte forma: a primeira se��o \ref{sec:aprendizado} � discutido sobre o processo  de aprendizado e o que isso representa. Na se��o \ref{sec:redesNeurais} � apresentado sobre redes neurais e as principais t�cnicas de redes neurais. Em \ref{sec:algoritmogenetico} � apresentado o algoritmo gen�tico (algoritmo evolutivo). Na se��o \ref{sec:neuroevolucao} � analisado o algoritmo de neuroevolu��o e apresentado um estudo sobre diferentes trabalhos que utilizam neuroevolu��o. Na se��o \ref{sec:aprendizadoComHumanos} � feito um estudo de pesquisas onde os humanos est�o diretamente ou indiretamente envolvidos com o processo de treinamento de uma rede neural artificial. Finalmente, na se��o \ref{sec:conclusaoAprendizadoJogosComHumanos}, s�o apresentadas as considera��o finais deste cap�tulo.

\section{Aprendizagem}
\label{sec:aprendizado}

Um dos processos mais naturais e comuns dos seres humanos � o processo de aprender. Segundo \cite{holt2012psychology} existem duas defini��es para aprendizagem que geralmente s�o usadas por psic�logos: "a mudan�a relativamente permanente no comportamento devido � experi�ncia passada"  ou, "o processo pelo qual ocorrem mudan�as relativamente permanentes no potencial comportamental como resultado da experi�ncia". J� \cite{de2013learning} afirma que tais defini��es cl�ssicas s�o problem�ticas e define o processo de aprendizado como adapta��o ontogen�tica, ou seja, como mudan�as no comportamento de um organismo resultam de regularidades no ambiente do organismo. Esta defini��o funcional n�o s� resolve os problemas de outras defini��es, mas tamb�m tem importantes vantagens para a pesquisa de aprendizagem cognitiva.

No trabalho de \cite{schunk1996learning}, a aprendizagem, na perspectiva filos�fica, pode ser discutida sob o t�tulo de epistemologia, que se refere ao estudo da origem, natureza, limites e m�todos de conhecimento. Ainda segundo o autor a origem do conhecimento � obtida de dois modos: 

\begin{itemize}

	\item \textbf{Racionalismo}. Refere-se � ideia de que o conhecimento � derivado da raz�o sem recorrer aos sentidos.
	
	\item \textbf{Empirismo}. No contraste ao racionalismo, o empirismo se refere a ideia que a experi�ncia � a fonte de conhecimento.

\end{itemize}

Fora do campo de psicologia e filosofia pode-se encontrar outras defini��es para o aprendizado, como no trabalho de \cite{carbonell1983overview} que define a aprendizagem como um fen�meno de m�ltiplas faces. O processo de aprendizado inclui a aquisi��o de novos conhecimentos, o desenvolvimento de habilidades motoras e cognitivas atrav�s de instru��es ou pr�ticas, a organiza��o do conhecimento, e a descoberta de novos fatos e teorias atrav�s da observa��o e experimenta��o.

\subsection{Aprendizado de m�quina}

O aprendizado de m�quina s�o modelos matem�ticos para representar o processo de aprendizado com m�quinas, entre outras palavras, o estudo do processo de aprendizado e a modelagem computacional desse processo definem o aprendizado de m�quina.

Segundo \cite{carbonell1983overview} o aprendizado de m�quina � organizado em tr�s amplos campos de pesquisa:

\begin{itemize}

	\item \textbf{Estudos orientados a tarefas}. Desenvolvimento e an�lise de sistemas de aprendizado em um predeterminado conjunto de tarefas.
	
	\item \textbf{Simula��o cognitiva}. Investiga��o e simula��o do processo de aprendizado humano.
	
	\item \textbf{An�lise te�rica}. Explora��o te�rica do espa�o dos poss�veis m�todos de aprendizado e algoritmos independentes do dom�nio da aplica��o.

\end{itemize}

Segundo \cite{russell1995modern} um agente que utiliza aprendizado de m�quina pode ser generalizado pelo seguinte diagrama da \ref{fig:modelLearningAgents}.

\begin{figure}[H]
\centering
\includegraphics[width=\columnwidth]{figures/modelLearningAgents.pdf}
\caption[Modelo de agente com aprendizagem]{Modelo de agente com aprendizagem.}
\label{fig:modelLearningAgents}
\end{figure}

O \textbf{elemento de aprendizado} � respons�vel por ajustar a intelig�ncia do agente. O elemento pega o conhecimento adquirido e um pouco do \textit{feedback} da performance do agente, e determina como o elemento de desempenho dever� ser modificado.

O \textbf{elemento de desempenho} avalia todas as entradas que recebe (sensores, aprendizado e gerador de problema) e escolhe a a��o.

O componente chamado \textbf{cr�tica} avalia qu�o bem o agente est� desempenhando. A cr�tica aplica um padr�o fixo de desempenho, isso � necess�rio porque as pr�prias percep��es n�o fornecem nenhuma indica��o do sucesso do agente.

Por �ltimo, o \textbf{gerador de problema} � respons�vel por sugerir a��es que podem levar a novas experi�ncias que tragam algo novo para o agente. Esse elemento faz com que o agente n�o fique transitando em um pequeno conjunto de a��es julgadas como boas, e avalie outras a��es que n�o foram exploradas ou que foram pouco exploradas.

� muito comum na literatura classificar aprendizado de m�quina de acordo com a avalia��o do \textit{feedback} que o agente faz. As classifica��es s�o:

\begin{itemize}

	\item \textbf{Aprendizagem supervisionada}. Em situa��o onde as entradas e sa�das do componente s�o conhecidas (o elemento que fornece esse mapeamento de sa�da e entrada � chamado de professor).
	
	\item \textbf{Aprendizagem n�o supervisionada}. Situa��o onde a sa�da correta n�o � conhecida .
	
	\item \textbf{Aprendizagem por refor�o}. Nessa abordagem cada a��o escolhida pelo agente recebe um refor�o ou puni��o sem nunca falar qual � a a��o correta ou a melhor a��o.

\end{itemize}

\section{Redes Neurais}
\label{sec:redesNeurais}

Segundo \cite{koehn1994combining} as redes neurais foram inventadas no esp�rito de ser uma met�fora biol�gica. A met�fora biol�gica para redes neurais � o c�rebro humano. Como o c�rebro, esse modelo computacional consiste em pequenas unidades interconectadas. Essas unidades (ou n�s) t�m habilidades bem simples. Assim, a for�a desse modelo deriva da intera��o dessas unidades. Ela depende da sua estrutura (topologia) e suas conex�es.

Uma rede neural � composta por um conjunto de n�s, ou \textbf{unidades}, conectadas por \textbf{liga��es}. Cada liga��o tem um valor num�rico chamado de \textbf{peso} associado a ele. Os pesos s�o o principal meio de armazenamento de longo prazo em redes neurais e, a aprendizagem geralmente ocorre atrav�s da atualiza��o desses pesos. Algumas dessas unidades s�o conectadas com o ambiente externo, e pode ser desenhada como unidades de entrada ou sa�da. Os pesos s�o modificados para tentar trazer a rela��o de entrada e sa�da da rede seja ajustada de acordo com o ambiente (\cite{russell1995modern}).

\subsection{Implementa��es de redes neurais}
\label{sec:implementacoesRN}

Essas pequenas unidades, conex�es e topologias podem ser comparadas a neur�nios e sinapses. Redes neurais de apenas uma camada s�o chamadas de \textit{perceptron}. Segundo \cite{beiu2003survey} uma rede neural de \textit{perceptron} � composta por um grafo onde os n�s s�o neur�nios e as arestas s�o as sinapses. Essa rede tem alguns n�s de entrada, e alguns (ao menos um) n�s de sa�da.

Pelo fato de o \textit{perceptron} ter apenas uma camada ele se torna muito limitado para resolver diferentes tipos de problemas, ou seja, o perceptron apenas resolve problemas linearmente separ�veis. O modelo \textit{perceptron} de m�ltiplas camadas (do ingl�s Multilayer Perceptron - MLP) apresenta uma ou mais camadas chamadas de unidades escondidas.

\subsubsection{Perceptron}

A imagem ilustra \ref{fig:perceptronModel} ilustra o funcionamento de um perceptron.

\begin{figure}[H]
\centering
\includegraphics[width=\columnwidth]{figures/perceptron.pdf}
\caption[Perceptron]{\textit{Perceptron}.}
\label{fig:perceptronModel}
\end{figure}

O modelo simples de \textit{perceptron} funciona com sa�das bin�rias, sendo muito utilizado para reconhecimento de padr�es, ele funciona da seguinte forma:

\begin{itemize}
	\item Um vetor de entrada X, onde cada posi��o do vetor representa uma caracter�stica diferente da entrada.
	\item Um vetor W chamado de vetor de pesos sin�pticos, para cada caracter�stica de entrada um peso � associado. Nesse vetor ocorrer�o ajustes de valor at� que a sa�da fique correta.
	\item Bias (b) � um valor de polariza��o que permite mudar a fun��o de ativa��o para a esquerda ou para a direita, ou seja, ela permite melhor espalhar os resultados que a fun��o de ativa��o gera esse valor tamb�m � ajustado durante o aprendizado.
\end{itemize}

A sa�da do \textit{perceptron} pode ser definida pela seguinte equa��o:

\begin{equation}
y = \sigma( \displaystyle\sum_{j=1}^{n} w_j x_ij + b_i ),
\end{equation}

onde \textbf{y} � o valor de sa�da(0 ou 1), \textbf{$\sigma$} � a fun��o de ativa��o. Retorna 1 para valores maiores que zero, e retorna 0 para valores menores ou iguais a zero e \textbf{n} � o tamanho do vetor de entrada.

At� que para todas as entradas, todas sa�das estejam corretas o vetor W e o \textit{bias} s�o atualizados. Segundo \cite{estebon1997perceptrons} ajustando os pesos das conex�es entre as camadas, a sa�da do \textit{perceptron} pode ser "treinada" para corresponder com a sa�da desejada. O treino � completo quando um conjunto de entradas passa pela rede e os resultados obtidos s�o os resultados desejados. Se existir alguma diferen�a entre a sa�da atual e a sa�da desejada, os pesos s�o ajustados para produzir um conjunto de sa�da mais pr�ximo aos valores desejados.

A equa��o de treinamento (ajuste de pesos) pode ser escrita da seguinte forma:

\begin{equation}
w'_ij = w_ij + \alpha(t_j - e_j) \times x_ij,
\end{equation}

onde \textbf{$w'_ij$} � o novo valor do peso, \textbf{$w_ij$} � o valor atual do peso, \textbf{$\alpha$} taxa de aprendizagem onde $0 \leq \alpha \leq 1$. Valores baixos para $\alpha$ faz com que os pesos sejam ajustados suavemente. A vari�vel \textbf{$t_j$} representa resultado esperado, \textbf{$e_j$} resultado atual e \textbf{$x_ij$} o valor de entrada.

O perceptron resolve qualquer problema linearmente separ�vel como calcular o \textbf{E} l�gico de duas entradas. A imagem \ref{fig:perceptronE} mostra um plano cartesiano onde conseguimos com uma reta separar entradas cujo valor esperado de sa�da � igual 0 das entradas que o valor sa�da � igual a 1.

\begin{figure}[H]
\centering
\includegraphics[width=10cm]{figures/perceptronE.pdf}
\caption[Perceptron E l�gico]{\textit{Perceptron} E l�gico.}
\label{fig:perceptronE}
\end{figure}

Em casos onde n�o o problema n�o � linearmente separ�vel o perceptron falha. Como mostra a imagem  \ref{fig:perceptronXOR} n�o � poss�vel separar linearmente o problema do ou exclusivo (XOR).

\begin{figure}[H]
\centering
\includegraphics[width=10cm]{figures/perceptronXOR.pdf}
\caption[Perceptron XOR]{\textit{Perceptron} XOR.}
\label{fig:perceptronXOR}
\end{figure}

\subsubsection{Perceptron de m�ltiplas camadas}

Um \textit{perceptron} de m�ltiplas camadas modifica um pouco o desenho de como funciona o \textit{perceptron}. A imagem \ref{fig:multilayerperceptron} mostra como funciona o MLP. Conforme mostra na imagem, o MLP tem uma camada de entrada, uma ou mais camadas escondidas (a quantidade de neur�nios nas camadas escondidas n�o precisa ser uniforme, ou seja, ter a mesma quantidade de neur�nios da camada de entrada/sa�da) e uma camada de sa�da.

\begin{figure}[H]
\centering
\includegraphics[width=\columnwidth]{figures/multilayerPerceptron.pdf}
\caption[Perceptron de m�ltiplas camadas]{\textit{Perceptron} de m�ltiplas camadas.}
\label{fig:multilayerperceptron}
\end{figure}

O MLP � uma rede fortemente conectada, ou seja, � uma rede onde as camadas est�o em certa ordem e os neur�nios de uma camada estimulam todos os neur�nios da camada subsequente. Essas conex�es s�o chamadas de \textit{feedfoward}.

O funcionamento do MLP � bem semelhante ao do \textit{perceptron}. O algoritmo de treino mais simples para MLP � chamado de retro-propaga��o. Essa equa��o pode ser escrita da seguinte maneira:

\begin{equation}
\delta w^k_ij = \eta \times \frac{\partial E}{\partial w^k_ij},
\end{equation}

onde \textbf{$k$} � a camada que est� sendo atualizada, \textbf{$\eta$} � a taxa de aprendizado, \textbf{$\partial$} representa a derivada parcial, \textbf{$E$} representa o valor de erro.

O valor de erro � representado pela seguinte equa��o:

\begin{equation}
E = \frac{1}{2} \times \displaystyle\sum_{i}{} (d_i - y_i)^2,
\end{equation}

onde \textbf{$d_i$} � o valor desejado de sa�da e \textbf{$y_i$} � a sa�da obtida.

� poss�vel resolver o problema do XOR utilizando um MLP simples com uma camada escondida. A ilustra��o \ref{fig:MLPXOR} mostra como ficaria a solu��o.

\begin{figure}[H]
\centering
\includegraphics[width=10cm]{figures/MLPXOR.pdf}
\caption[MLP XOR]{\textit{MLP} XOR.}
\label{fig:MLPXOR}
\end{figure}

\section{Algoritmos gen�ticos}
\label{sec:algoritmogenetico}

A primeira pesquisa na �rea de algoritmos gen�ticos foi feito por John Holland no livro \textit{Adaptation in Natural and Artificial Systems}. Segundo \cite{mitchell1995genetic} algoritmos gen�ticos (GA) � uma abstra��o da evolu��o biol�gica, onde se move uma popula��o de cromossomos (representando candidatos a solu��es de um determinado problema) para uma nova popula��o, usando "sele��o"  junto com operadores baseados em gen�ticas para cruzamentos e muta��o. 

Segundo \cite{485447} antes de aplicar o algoritmo gen�tico, � necess�rio mapear qual ser� a arquitetura do cromossomo de modo que ele represente uma solu��o para o problema. Para usar o modo convencional do algoritmo gen�tico com cromossomos de tamanhos fixos, � preciso:

\begin{itemize}

	\item Determinar o esquema de representa��o.
	
	\item Determinar como mensurar o \textit{fitness}. Segundo \cite{mitchell1995genetic} o \textit{fitness} � definido por uma fun��o que atribui uma nota (\textit{fitness}) para cada cromossomo da popula��o atual. O \textit{fitness} do cromossomo depende de qu�o bem o cromossomo resolve o problema.
	
	\item Determinar os par�metros e vari�veis para controlar o algoritmo.
	
	\item Determinar um modo de mostrar o resultado e um crit�rio de parada para o algoritmo.
	
\end{itemize}

Segundo \cite{485447} o algoritmo evolutivo pode ser separado em tr�s passos:

\begin{itemize}
	
	\item Aleatoriamente criar uma popula��o inicial de cromossomos (solu��es para o problema).
	
	\item Executar os seguintes sub-passos na popula��o at� que o crit�rio de parada seja satisfeito:
		
		\subitem \textbf{1:} Determinar para cada indiv�duo um \textit{fitness} atrav�s da fun��o avaliadora de \textit{fitness}.
		
		\subitem \textbf{2:} Selecionar quais indiv�duos passar�o para a pr�xima gera��o por uma probabilidade baseada em seu \textit{fitness}. Aplicar na nova popula��o de indiv�duos as 3 seguintes opera��es gen�ticas: copiar um indiv�duo para a nova popula��o; criar dois novos indiv�duos combinando seus cromossomos atrav�s de alguma opera��o de cruzamento (\textit{crossover}); fazer a muta��o de algumas caracter�sticas de um cromossomo aleatoriamente.
	
	\item Pegar o melhor indiv�duo j� criado (aquele com melhor \textit{fitness}) para utilizar na resolu��o do problema.
	
\end{itemize}
 
O resultado de algoritmo gen�tico muitas vezes n�o � o melhor resultado poss�vel, por�m gera um resultado excelente em cen�rios onde n�o se sabe como resolver um problema, ou em cen�rios onde a solu��o �tima demora muito tempo para ser calculada.

\section{Neuroevolu��o}
\label{sec:neuroevolucao}


No trabalho \textit{Combining genetic algorithms and neural networks: The encoding problem} de \cite{koehn1994combining} � levantado a seguinte quest�o: Se ambas as t�cnicas s�o aut�nomas, porque ent�o combin�-las? Ainda segundo \cite{koehn1994combining} a resposta curta para essa quest�o diz que o problema de redes neurais � o n�mero de par�metros que precisam ser atribu�dos antes que qualquer treino come�e, nesse ponto entra o algoritmo gen�tico. O autor do trabalho completa dizendo que a inspira��o vem da natureza, o sucesso de um indiv�duo n�o � determinado apenas pelo conhecimento e habilidades que ele ganha atrav�s da experi�ncia, tamb�m depende de sua heran�a gen�tica.

O trabalho de \cite{risi2014neuroevolution} define o algoritmo b�sico de neuroevolu��o da seguinte forma: uma popula��o de gen�tipos que codificam redes neurais artificiais � evolu�da para encontrar uma rede (pesos e/ou topologia) que resolvam um problema computacional. Normalmente cada gen�tipo � codificado em uma rede neural, que ent�o � testada para uma espec�fica tarefa por um determinado per�odo de tempo. O desempenho ou \textit{fitness} da rede � guardado e, uma vez que os valores de \textit{fitness} para os gen�tipos da popula��o s�o determinados, uma nova popula��o � gerada pela altera��o ligeira dos gen�tipos que codificam a rede neural artificial (muta��o) ou pela combina��o de v�rios gen�tipos (\textit{crossover}). Em geral, gen�tipos com alto \textit{fitness} tem uma alta chance de ser selecionado para reprodu��o e os seus descendentes substituem gen�tipos com valores de \textit{fitness} mais baixos, formando assim uma nova gera��o. 

Segundo \cite{miikkulainen2006computational} existem v�rios m�todos para a evolu��o das redes neurais. A maneira mais direta � formar a codifica��o gen�tica para a rede pela concatena��o dos valores num�ricos de seus pesos, e evoluir uma popula��o de tais codifica��es usando \textit{crossover} e muta��o. 

Embora esta abordagem padr�o seja f�cil de implementar e pr�tica em muitos dom�nios, m�todos mais sofisticados podem resolver problemas muito mais dif�ceis. Dois m�todos bem comuns para evoluir redes neurais artificiais s�o: subpopula��es  for�adas (do ingl�s Enforced Sub-Populations - ESP) e neuroevolu��o de topologias de aumento (do ingl�s Neuroevolution of Augmenting Topologies - NEAT).
 
\subsection{ESP}

No ESP \cite{gomez1997incremental} a popula��o consiste de neur�nios individuais ao inv�s de redes completas, e um subconjunto de neur�nios � reunido para formar uma rede completa. No entanto, o ESP aloca uma popula��o separada para cada uma das unidades na rede, e um neur�nio s� pode ser recombinado com membros de sua pr�pria subpopula��o. Entre outras palavras, o foco dessa abordagem � a evolu��o dos neur�nios ao inv�s das redes.

A imagem \ref{fig:espModel} esquematiza o modelo ESP. A popula��o de neur�nios � reunida em subpopula��es  mostrada na ilustra��o como grandes c�rculos. E finalmente a rede neural � formada escolhendo aleatoriamente um neur�nio de cada subpopula��o.

\begin{figure}[H]
\centering
\includegraphics[width=\columnwidth]{figures/ESP.pdf}
\caption[Modelo ESP]{Modelo \textit{ESP}.}
\label{fig:espModel}
\end{figure}


\subsection{NEAT}
\label{sec:neat}

O NEAT baseia-se na ideia de evoluir a topologia de rede, ou seja, adicionando novos neur�nios, camadas e conex�es na rede neural artificial. Os autores \cite{stanley2003evolving} dizem que o espa�o de par�metros pesquisado no espa�o de redes adaptativas (como as redes geradas pelo NEAT) � muito maior do que para redes est�ticas (topologias que nunca mudam). Al�m dos pesos de conex�o, v�rios par�metros de aprendizado devem ser evolu�dos para cada regra de aprendizado em cada conex�o. Assim, � importante minimizar o n�mero de conex�es otimizadas pela evolu��o. Al�m disso, a topologia tamb�m precisa ser optimizada. � necess�rio ter conex�es entre os n�s \textit{corretos}, para que as conex�es possam ser fortalecidas ou enfraquecidas para alterar a rela��o entre os conceitos computacionais que elas representam.


A imagem \ref{fig:neatModel} esquematiza o modelo NEAT. No exemplo cada gene cont�m uma conex�o de neur�nios (X, Y se refere a conex�o do neur�nio X com o Y), al�m da conex�o � guardado o peso, o n�mero acima do gene � chamado de \textbf{n�mero de inova��o}. Os genes com fundos de cor mais escura mostra que ele est� desabilitado. 

Ainda na imagem \ref{fig:neatModel}, � mostrado 2 evolu��es, a evolu��o da parte superior da imagem acontece adicionando uma nova conex�o (n�mero de inova��o 7), e a parte inferior da imagem mostra a adi��o de um novo n� (desabilitando o gene de n�mero de inova��o 3, e adicionando gene de n�mero de inova��o 7 e 8).

\begin{figure}[H]
\centering
\includegraphics[width=\columnwidth]{figures/NEAT.pdf}
\caption[Modelo NEAT]{Modelo \textit{NEAT}.}
\label{fig:neatModel}
\end{figure}

\subsection{Vantagens e Desvantagens}

Segundo \cite{risi2014neuroevolution} existe uma s�rie de motivos para usar a neuroevolu��o e alguns motivos onde utilizar neuroevolu��o n�o � t�o interessante. As principais vantagens de utiliz�-la �:

\begin{itemize}

	\item \textbf{Desempenho para bater recordes}. Para alguns problemas, neuroevolu��o simplesmente fornece o melhor desempenho em concorr�ncia com outros m�todos de aprendizagem ("desempenho" �, naturalmente, definido de forma variada para diferentes problemas).
	
	\item \textbf{Ampla aplicabilidade}. Outro benef�cio da neuroevolu��o � que ela pode ser usada para tarefas de aprendizagem supervisionadas, n�o supervisionadas e refor�adas. Neuroevolu��o s� requer algum tipo de avalia��o num�rica da qualidade das suas redes.

	\item \textbf{Escalabilidade}. Comparado com muitos outros tipos de aprendizado por refor�o, neuroevolu��o parece lidar muito bem com grandes espa�os de a��o/estado.
	
	\item \textbf{Diversidade}. A neuroevolu��o pode se basear na rica fam�lia de m�todos de preserva��o da diversidade e m�todos multiobjectivos que foram desenvolvidos dentro da comunidade de computa��o evolutiva.
	
	\item \textbf{Aprendizagem aberta}. Enquanto neuroevolu��o pode ser usado para aprendizado por refor�o, pode-se argumentar que poderia ir al�m desta formula��o relativamente restrita de aprendizagem de refor�o. Em particular, nos casos em que a topologia da rede tamb�m evolui, a neuroevolu��o poderia, em princ�pio, apoiar a evolu��o aberta, onde o comportamento de complexidade arbitr�ria e sofistica��o poderiam emergir. Concretamente, os algoritmos de neuroevolu��o costumam buscar em um espa�o muito grande.
	
	\item \textbf{Permitir novos tipos de jogos}. Em jogos que o jogador humano pode evoluir os seus personagens, os equipamentos de seus personagens, seu time e outras customiza��es, seria muito dif�cil resolver com os tradicionais m�todos de aprendizado. Computa��o evolutiva aqui fornece qualidades exclusivas para o \textit{design} do jogo e, alguns projetos dependem especificamente da neuroevolu��o.
	
\end{itemize}

Em contrapartida tamb�m existe situa��es que neuroevolu��o n�o � t�o interessante, seja pelos resultados esperados ou por ter outras t�cnicas que teriam melhor desempenho. O principal problema apontado por \cite{risi2014neuroevolution} � que, as redes neurais evolu�das tendem a ter caracter�sticas de "caixa preta", o que significa que um humano n�o pode facilmente descobrir o que ela faz e como ela funciona olhando para elas. Em jogos comerciais, por exemplo, seria muito dif�cil predizer e/ou depurar uma determinada escolha de um agente que utiliza neuroevolu��o.

\subsection{Neuroevolu��o em jogos}

Nesta subse��o vamos ver um pouco de como foi implementado a neuroevolu��o em diferentes jogos, especificamente vamos analisar os jogos digitais: The Open Racing Car Simulator ou TORCS \cite{cardamone2009line}, Galatic Arms Race ou GAR \cite{hastings2009automatic}, Creatures \cite{grand1997creatures}, Neuroevolving Robotic Operatives ou NERO \cite{stanley2005real} e, por fim, neuroevolu��o aplicado ao jogo de tabuleiro Reversi \cite{evoltuinaryneuralnetworks}.

\subsubsection{Neuroevolu��o no jogo TORCS}

Come�ando pelo jogo TORCS \cite{cardamone2009line}, ele � um jogo digital de corrida de carros em tr�s dimens�es de c�digo fonte aberto. Ele utiliza a t�cnica de neuroevolu��o NEAT detalhada em \ref{sec:neat}. A rede neural tem acesso a uma s�rie de informa��es cedidas por sensores, entre elas: 

\begin{itemize}
	\item \textbf{Sensores das bordas da pista}. Existem seis sensores laterais que marcam a dist�ncia para os limites da pista. Em rela��o ao carro esses sensores est�o localizados a: $-90^{\circ}$, $-60^{\circ}$, $30^{\circ}$, $60^{\circ}$ e $90^{\circ}$. 
	
	\item \textbf{Sensores agregados}. Existem tr�s sensores frontais (que simulariam a vis�o perif�rica do motorista) que tamb�m marcam a dist�ncia para os limites da pista: $-10^{\circ}$, $0^{\circ}$ e $10^{\circ}$. Em algumas situa��es como em grandes retas esses sensores podem n�o conseguir estimar a dist�ncia.
	
	\item \textbf{Velocidade}. A velocidade que o ve�culo est� se movendo.
\end{itemize}

A imagem \ref{fig:sensoresTorcs} de \cite{cardamone2009line} mostra como os sensores de dist�ncia funcionam.

\begin{figure}[H]
\centering
\includegraphics[width=10cm]{figures/sensoresTORCS.png}
\caption[Sensores TORCS]{Sensores do jogo \textit{Torcs} \cite{cardamone2009line}.}
\label{fig:sensoresTorcs}
\end{figure}

Para fazer o aprendizado das redes neurais todas elas come�avam sem conhecimento algum e, as redes neurais n�o tinham camadas escondidas, ou seja, a camada de entrada era diretamente conectada a camada de sa�da. O tamanho da popula��o era de 100 indiv�duos que corriam 10 vezes em uma mesma pista fazendo 2000 voltas na pista. Como mostra a imagem \ref{fig:resultsTorcs} de \cite{cardamone2009line} as redes j� tinham excelentes resultados ap�s 300 voltas (ainda na imagem � comparado o desempenho de 3 t�cnicas diferentes: softmax, e-greedy e offiline).

\begin{figure}[H]
\centering
\includegraphics[width=10cm]{figures/resultsTorcs.png}
\caption[Desempenho Neuroevolu��o TORCS]{Desempenho da neuroevolu��o no TORCS.}
\label{fig:resultsTorcs}
\end{figure}

Um dos problemas relatados pelos autores � avalia��o de \textit{fitness}, como o agente estava trabalhando com o jogo em funcionamento n�o era poss�vel fazer c�lculos muito demorados para n�o comprometer o andamento da volta. Como o \textit{fitness} s� era avaliado no final da volta, trechos muito bem feitos dentro de uma volta muito lenta podiam ser descartados.

\subsubsection{Neuroevolu��o no jogo GAR}

Outro estudo de neuroevolu��o em jogos � estudo de \cite{hastings2009automatic} com o jogo GAR. O jogo consiste em um batalha de naves com grande variedade de armas e cen�rios. A principal abordagem dessa pesquisa � que nesse jogo a neuroevolu��o foi usada para cria��o de conte�do autom�tico ao inv�s de controlar um personagem ou objeto. A principal ideia � estender a experi�ncia de jogo do usu�rio criando novos conte�dos baseado nas prefer�ncias dos jogadores.

O GAR utiliza um agente chamado cgNEAT (do ingl�s Content Generation Neuroevolution of Augmenting Topologies) com uma varia��o de redes neurais artificias chamadas de CPPN (do ingl�s Compositional Pattern Producing Networks) que difere principalmente no seu conjunto de fun��es ativadoras e como elas s�o aplicadas.

A premissa para avalia��o de \textit{fitness} das CPPNs � que os usu�rios experimentam conte�dos novos e, ap�s experimentar pela primeira vez, voltam a visitar esse conte�do caso gostem e, caso n�o gostem, visitam muito pouco ou n�o visitam mais esses conte�dos.

O algoritmo de evolu��o � separado em sete passos:

\begin{itemize}
	\item \textbf{1}. Cada novo conte�do � representado por uma CPPN. Diferentes tipos de conte�dos (armas, cen�rios, entre outros) podem ser representados por diferentes tipos de CPPPN (configura��es de entrada/sa�da diferentes).
	
	\item \textbf{2}. Durante o jogo, para cada conte�do � atribu�do um valor de \textit{fitness} baseado no qu�o frequente os jogadores escolhem ou usam aquele conte�do.
	
	\item \textbf{3}. Jogadores podem come�ar com conte�do aleat�rio ou com um conte�do chamado de conjunto inicial predefinido.
	
	\item \textbf{4}. Conte�dos s�o espalhados pelo mundo. Esses conte�dos s� estar�o eleg�veis para reprodu��o somente se jogadores o testarem.
	
	\item \textbf{5}. O conte�do � reproduzido pelo cgNEAT da seguinte maneira: o algoritmo escolhe alguns conte�dos gerados pelo passo 4. Esses conte�dos s�o selecionados por m�todo de roleta e, a muta��o e o cruzamento acontecem de acordo com o algoritmo de NEAT.
	
	\item \textbf{6}. Para qualquer novo conte�do gerado, existe uma probabilidade (selecionada pelo designer) de ser escolhida de um conjunto de desova, que � uma cole��o de conte�do pr�-evolu�do, em vez de ser reproduzida dos pais. Esta piscina garante que a diversidade n�o � perdida e que os bons tipos de conte�dos do passado possam reaparecer.
	
	\item \textbf{7}. Conte�dos com \textit{fitness} muito alto (muito popular entre os jogadores) podem ser salvo como conte�do arquivado.
	
\end{itemize}

\subsubsection{Neuroevolu��o no jogo Creatures}

No jogo Creatures \cite{grand1997creatures} existem agentes que s�o bichos de estima��o virtuais que s�o chamados de criaturas. A arquitetura das criaturas � baseada na biologia animal. Cada criatura tem uma rede neural respons�vel pela coordena��o sens�rio-motora e sele��o de comportamento, e uma "bioqu�mica artificial" que modela um metabolismo energ�tico simples.

Cada rede neural � subdividia em l�bulos que definem as caracter�sticas el�tricas, qu�micas e morfol�gicas de um grupo de c�lulas. Ao todo cada rede neural tem aproximadamente 1000 neur�nios agrupados em 9 l�bulos e interconectados por 5000 sinapses. A imagem de \ref{fig:creatureNeurons} de \cite{grand1997creatures} mostra como est�o dispostos esses elementos.

\begin{figure}[H]
\centering
\includegraphics[width=10cm]{figures/CreaturesNeurons.png}
\caption[Rede neural Creatures]{Rede neural Creatures.}
\label{fig:creatureNeurons}
\end{figure}

A evolu��o nesse se jogo acontece no nascimento de uma nova criatura, ou seja, quando duas criaturas cruzam. O genoma forma um �nico cromossoma haploide. Durante a reprodu��o, os genes parentais s�o cruzados e emendados nos limites dos genes. Erros ocasionais de cruzamento  podem introduzir omiss�es de genes e duplica��es. Um pequeno n�mero de muta��es aleat�rias para os corpos gen�ticos � tamb�m aplicado. Para evitar uma taxa de falha excessiva devido a erros de reprodu��o em genes cr�ticos, cada gene � precedido por um cabe�alho que especifica que opera��es (omiss�o, duplica��o e muta��o) podem ser realizadas sobre ele. O cruzamento � realizado de tal maneira que a liga��o do gene � proporcional � dist�ncia de separa��o, permitindo caracter�sticas ligadas como seria de esperar (por exemplo, temperamento da criatura com tipo facial). Como o genoma � haploide, temos que evitar que as caracter�sticas �teis ligadas ao sexo sejam erradicadas simplesmente porque elas foram herdadas por uma criatura do sexo oposto. Portanto, cada gene cont�m as instru��es gen�ticas para ambos os sexos, mas apenas os genes n�o sexuados e apropriadamente sexuados s�o expressos no fen�tipo.

\subsubsection{Neuroevolu��o no jogo NERO}

Para os autores \cite{stanley2005real} um dos grandes problemas dos jogos � que, em maioria deles a intelig�ncia aplicada � feito atrav�s de um algoritmo est�tico, ou seja, uma vez que o jogador descobre uma fraqueza de um inimigo, n�o importa quantas vezes o jogador explore a fraqueza ela sempre ir� existir.

No trabalho de \cite{stanley2005real} com o jogo NERO a neuroevolu��o tem papel muito interessante. Nesse estudo � apresentado o jogo NERO, onde o jogador tem um ex�rcito de rob�s e com esse ex�rcito tem que enfrentar uma s�rie de desafios. No NERO a neuroevolu��o � aplicada no ex�rcito, ou seja, o jogador humano pode treinar seus rob�s para que eles evoluam durante o jogo.

O NERO utiliza uma varia��o de NEAT para ambientes de tempo real, chamado pelos autores de rtNEAT (do ingl�s Real-time NEAT). O modelo do NEAT empregado � semelhante a figura \ref{fig:neatModel}.

O algoritmo empregado funciona da seguinte forma:

\begin{itemize}
	\item \textbf{1}. Calcular o \textit{fitness} ajustado: $f_i$ representa \textit{fitness} original do indiv�duo $i$ e, $|S|$ a quantidade de indiv�duos na popula��o. O \textit{fitness} ajustado dele � dado por: $\frac{f_i}{|S|}$.
	
	\item \textbf{2}. Removendo o pior agente: O objetivo desta etapa � remover um agente de desempenho muito abaixo do jogo, esperan�osamente para ser substitu�do por algo melhor. O agente deve ser cuidadosamente escolhido para preservar a din�mica de especia��o. Se o agente com a maior \textit{fitness} n�o ajustado fosse escolhido, o compartilhamento do \textit{fitness} n�o poderia mais proteger a inova��o porque novas topologias seriam removidas assim que aparecerem.
	
	\item \textbf{3}. Criando descendentes: os descendentes s�o criados um por vez e, a probabilidade de escolher uma dada esp�cie progenitora � proporcional � seu \textit{fitness} ajustado em compara��o com a m�dia de \textit{fitness} total de todas as esp�cies.
	
	\item \textbf{4}. Atribuir novamente agentes a esp�cies: O limiar de compatibilidade din�mica mant�m o n�mero de esp�cies relativamente est�vel ao longo da evolu��o. Essa estabilidade � particularmente importante em um jogo de tempo real, uma vez que a popula��o pode precisar ser consistentemente pequena para acomodar recursos de CPU/GPU dedicados ao processamento gr�fico.
	
	\item \textbf{5}. Substituir um agente velho por um agente novo: uma vez que um indiv�duo foi removido no passo 2, a nova prole precisa substitu�-lo. Como os agentes s�o substitu�dos depende do jogo. Em alguns jogos (como o NERO), a rede neural pode ser removida de um corpo e substitu�da sem fazer nada ao corpo. Em outros, o corpo pode ter sido destru�do e precisa ser substitu�do tamb�m. O algoritmo rtNEAT pode trabalhar com qualquer um desses esquemas, desde que uma antiga rede neural seja substitu�da por uma nova.
	
\end{itemize}

Outra contribui��o muito interessante destacada pelos autores � a possibilidade de abrir novas possiblidades de tipos de jogos, onde o jogador customiza e treina seu time para realizar desafios ou para duelar com times treinados por outras pessoas.

\subsubsection{Neuroevolu��o no jogo Reversi}
\label{sec:neuroReversi}

No trabalho de \cite{evoltuinaryneuralnetworks} foi criado uma rede neuroevolutiva para jogar Reversi (jogo de tabuleiro). Diferentes dos outros exemplos nesse estudo a neuroevolu��o foi utilizada em conjunto com busca em �rvore de decis�es (minimax e alpha-beta). 

Nesse estudo as redes neurais s�o chamadas de redes de foco. Dado uma �rvore de escolhas essas redes definem \textbf{janelas} dentro dessa �rvore, ou seja, dentro da �rvore ela define quais n�s s�o interessantes de ser explorados. Na figura \ref{fig:focusedMinimax} as janelas s�o definidas pela �rea com o fundo mais escuro.

\begin{figure}[H]
\centering
\includegraphics[width=12cm]{figures/focusedMinimax.png}
\caption[Janelas dentro da �rvore de decis�o]{Janelas dentro da �rvore de decis�o.}
\label{fig:focusedMinimax}
\end{figure}

No algoritmo apenas n�s dentro de janelas s�o avaliados. Outra vantagem dessa abordagem � que ela torna a tomada de decis�o de um agente mais org�nica, ou seja, algoritmos comuns de busca em �rvore podem muitas vezes escolher jogadas n�o muito comum para jogadores humanos uma vez que ele est� analisando um grande n�mero de poss�veis pr�ximas jogadas.

Os autores enumeram duas grandes vantagens de utilizar essa t�cnica: 

\begin{itemize}
	
	\item Reduzir o n�mero de n�s avaliados: O fator de ramifica��o � reduzido o que acelera consideravelmente a procura. Com esse resultado, a busca pode fazer pesquisas mais profundas nas �rvores em caminhos mais promissores.
	
	\item As redes de foco s�o for�adas a decidir quais movimentos a pesquisa minimax deve avaliar e, a fim de jogar bem, eles devem desenvolver uma compreens�o do algoritmo minimax. � poss�vel que eles tamb�m descubram as limita��es do minimax e da fun��o de avalia��o, e aprendam a compensar n�o permitindo que o minimax veja certos movimentos.

\end{itemize}

O gr�fico \ref{fig:focusedNetworkdsResults} mostra o resultado obtido de um minimax com rede de foco comparado a um simples minimax. O gr�fico mostra o resultado analisando �rvores que analisavam at� 6 n�veis de profundidade. Para coletar esses dados foram jogados 244 jogos ap�s evoluir a rede de foco.

\begin{figure}[H]
\centering
\includegraphics[width=12cm]{figures/focusedNetworksResults.png}
\caption[Resultados da rede de foco]{Resultados da rede de foco com minimax contra um simples minimax.}
\label{fig:focusedNetworkdsResults}
\end{figure}

Para chegar nesse ponto, foi implementado uma popula��o de 50 redes neurais. A rede neural era empregado uma �rvore de busca minimax analisando n�s com profundidade at� 2. Seu oponente no treinamento utilizava a t�cnica \textit{alpha-beta} (apresentada em \ref{sec:podaAlphaBeta}) avaliando todos os n�s at� o terceiro n�vel de profundidade (um a mais que a rede utilizando neuroevolu��o). Para avalia��o de heur�stica, ambos os jogadores utilizam a mesma t�cnica baseada na estrat�gia posicional de \cite{rosenbloom1982world}. O \textit{fitness} de cada indiv�duo era baseado na percentagem de vit�rias ap�s 244 jogos entre essas redes. Essas redes evolu�ram por 1000 gera��es que levou cerca de uma semana.

\section{Treinamento com humanos}
\label{sec:aprendizadoComHumanos}

At� o momento vimos sobre redes neurais, t�cnicas, utilidade e exemplos de utiliza��o. Outro segmento importante de pesquisa dentro de rede neural � a entradas de dados por meio de jogadores humanos, ou seja, estudo de t�cnicas e ambientes que permitam que redes neurais tenham o aprendizado feito totalmente ou guiado por humanos.

Os autores \cite{karpov2015evaluating} em sua pesquisa, tamb�m utilizando o jogo NERO, avaliaram a influ�ncia e a import�ncia de um ambiente de aprendizagem guiado por humanos. No ambiente proposto, cada humano poderia treinar seu ex�rcito de rob�s especificando os desafios que o agente teria que realizar. 

Ainda sobre o trabalho de \cite{karpov2015evaluating} foi realizado um torneio entre ex�rcitos treinados por diferentes jogadores e os autores concluem que os resultados demonstram que os comportamentos bem sucedidos em um complexo jogo multi-agente podem ser altamente diversos, que � exatamente o que torna o treinamento dessas equipes um componente interessante do jogo. Os resultados tamb�m mostram que comportamentos complexos podem ser constru�dos utilizando neuroevolu��o, demonstrando seu poder como uma abordagem para a constru��o de NPCs para jogos complexos. Finalmente, esse torneio e os dados gerados s�o uma ferramenta experimental promissora no estudo de m�todos de aprendizagem de m�quinas guiadas por humanos. Em geral, os jogos de aprendizagem de m�quina s�o um g�nero vi�vel para desenvolvedores de jogos e uma ferramenta �til para pesquisadores de intelig�ncia artificial.

\subsection{Agentes Jogadores}

Um dos modos mais comuns de se aplicar intelig�ncia artificial em jogos � para cria��o de agentes jogadores (tamb�m chamado de \textit{bots}). Esses agentes podem ter como objetivo: ganhar de todos os jogadores, criar desafios mais adaptados aos jogadores e imitar jogadores humanos. Nessa subse��o ser� apresentado estudos onde os humanos influem no treinamento dos agentes.

Um famoso agente que utiliza rede neural e que treina com dados de humanos � o AlphaGo \cite{silver2016mastering}. O agente treina com uma base de dados coletados de jogadores profissionais com mais de 30 milh�es de jogadas. Segundo os autores do artigo o agente pode predizer corretamente o movimento de um jogador profissional com uma precis�o de 57\% de acerto.

O jogo Forza Motorsport da Microsoft tem integrado um m�dulo chamado Drivatar \cite{microsoft:drivatar}. Por ser um jogo comercial n�o existem muitos detalhes de como ele funciona. Basicamente, esse m�dulo cria um agente jogador com as mesmas caracter�sticas que o jogador humano e, esse agente � compartilhando entre seus amigos para integrar esse avatar no jogo deles e vice-versa. O jogo utiliza uma rede neural para cada avatar e, segundo \cite{ign:drivatar} o Drivatar est� preocupado principalmente como o jogador se comporta em momentos chaves de uma corrida, como por exemplo: em uma ultrapassagem ou em uma curva muito acentuada. Caracter�sticas como o qu�o r�pido o jogador vira o volante e o qu�o forte ele pisa nos aceleradores e freios tamb�m s�o analisados para gera��o desses agentes.

\subsubsection{Teste de Turing para bots}
\label{sec:testeTuringBots}

Outro ramo onde a neuroevolu��o pode ser empregada � no chamado Teste de Turing para agentes jogadores \cite{hingston2009turing} que, entre outras palavras, testa a habilidade de agentes (\textit{bots}) de imitar comportamento humano. 

Esse teste se tornou um desafio em 2008 com o jogo Unreal Tournament. Em cada partida um n�mero igual de jogadores e \textit{bots} competem em times. Cada jogador humano recebe um equipamento chamado de arma do julgamento (do ingl�s Judging Gun) que � usado para marcar um personagem como \textit{bot} ou humano. Sempre que um jogador acerta o julgamento os humanos recebem 10 pontos e o \textit{bot} � eliminado da partida. Se o palpite estiver errado eles perdem 10 pontos e o humano � eliminado da partida. Os dois primeiros agentes considerados vencedores dessa competi��o foi o agente UT\textasciicircum2 proposto por \cite{schrum2011ut} e o agente MirrorBot proposto por \cite{polceanu2013mirrorbot}.

O agente UT2 segmentou o comportamento do agente em m�dulos que modelavam vest�gios humanos, pegar uma arma jogada no ch�o, pegar itens, disparar a arma de julgamento, observar, etc. A maioria dos m�dulos � executado como uma reprodu��o baseada em regras de a��es humanas reais gravadas de diferentes partidas. O m�dulo de batalha foi treinado atrav�s do uso de uma neuroevolu��o multiobjetiva combinando algoritmos NSGA-II (\cite{deb2000fast}) e NEAT, maximizando o dano causado e minimizando danos recebidos e colis�es com objetos (humanos dificilmente colidem com objetos de cen�rio).

O agente MirrorBot implementa dois m�dulos: padr�o e espelho. O m�dulo padr�o � um m�dulo de base de regras que permite ao agente navegar pelo mapa, julgar um determinado jogador como sendo humano ou bot e atacar jogadores inimigos. Sempre que o agente encontra um jogador amig�vel, ele troca para o modo de espelho onde ele vai come�ar a gravar os movimentos do outro jogador e ele vai replicar de volta depois de um curto atraso.

\subsection{Cria��o de conte�do}

Cria��o de conte�do pode trazer v�rios aspectos positivos para um jogo. Aumentar o fator de \textit{replay} (vontade que o jogador tem de jogar novamente um jogo), estender a vida �til do jogo criando mais mapas, desafios e equipamentos entre outros benef�cios.

O trabalho de \cite{hastings2009automatic} exemplifica a cria��o de conte�do utilizando como entradas dados de humanos. O agente foi desenvolvido para criar novas armas para as naves de combate do jogo GAR. Um fator interessante desse agente � que, todo o treinamento � feito em tempo real enquanto o jogo est� funcionando. Algumas caracter�sticas faz o treinamento das redes serem bem diferentes do convencional. Armas geradas n�o s�o descartadas a n�o ser que o jogador fa�a isso manualmente, o jogador n�o informa diretamente se gostou da arma, isso � avaliado conforme a utiliza��o de acordo com as op��es do usu�rio.

\subsection{Outros}

T�cnicas como neuroevolu��o podem permitir novos tipos de jogo como sugerem os autores \cite{stanley2005real}. No jogo NERO o treinamento � feito em um modo de jogo separado chamado treinamento. Nesse modo de jogo o jogador humano posicionava seu ex�rcito em um campo e definia alguns objetivos para seus rob�s completarem, a figura \ref{fig:NEROObjectives} mostra como o jogador humano controla os objetivos do treinamento. Como pode ser visto na imagem, podem ser definidos um ou mais objetivos para o ex�rcito e, o objetivo pode ser fazer algo ou n�o fazer algo (como por exemplo, se aproximar e n�o se aproximar), isso � indicado pela cor da barra (verde ou vermelha). Da esquerda para direita na imagem os objetivos s�o: aproximar dos inimigos, atacar os inimigos, evitar tomar tiro, mover-se para uma posi��o, ficar agrupado e ficar em p�.

\begin{figure}[H]
\centering
\includegraphics[width=12cm]{figures/NEROObjectives.png}
\caption[Objetivos de treino NERO]{Objetivos de treino NERO.}
\label{fig:NEROObjectives}
\end{figure}

Ap�s o treinamento as redes eram salvas e durante o modo batalha seu ex�rcito era colocado em diferentes desafios e em diferentes tipos de mapa e o jogador n�o podia atuar sobre os rob�s.

\section{Considera��es finais}
\label{sec:conclusaoAprendizadoJogosComHumanos}

Nesse cap�tulo foi apresentado um dos ramos mais conhecidos dentro de Intelig�ncia Artificial, o aprendizado de m�quina. Tamb�m foram mostradas aplica��es de aprendizado de m�quina envolvendo humanos.

O termo aprendizado transita entre diversas �reas de pesquisa, foram mostradas algumas defini��es sobre a perspectiva da psicologia, filosofia e computa��o.

Em seguida, foi mostrado como esse processo humano foi modelado matematicamente e aplicado em algoritmos de computador como, por exemplo, o perceptron.

Na sequ�ncia foi apresentado sobre evolu��o de esp�cies, outro processo biol�gico e seu modelo matem�tico aplicado � computa��o.

Utilizando desses dois processos (aprendizagem e evolu��o) surge uma nova t�cnica chamada neuroevolu��o. Foi apresentado seu funcionamento, algumas t�cnicas, suas principais caracter�sticas, vantagens e desvantagens e foi mostrado a sua versatilidade analisando a aplica��o dela em diferentes tipos de problemas. Entre os diferentes modos de como a neuroevolu��o foi empregado vimos: cria��o de agente jogador, cria��o de novo conte�do, cria��o de personagem e cria��o de estrat�gias de jogo.

Por fim foram analisados diversos estudos onde os humanos est�o envolvidos dentro do aprendizado de uma rede neural. Nesses estudos podemos ver alguns modos indiretos de usar humanos, como no AlphaGo onde os dados humanos est�o formatados como base de dados, e algumas aplica��es mais diretas e em tempo real, como no jogo GAR onde o pr�prio jogo gera conte�do novo de acordo com a aceita��o dos jogadores.
	\chapter{MCTS para o jogo Pok�mon}
\label{chap:mctsPokemon}

\section{Introdu��o}


\section{Tomada de decis�o}
\label{sec:tomadaDecisao}

%	\chapter{Conclus�es e pesquisas futuras}
\label{chap:conclusoesPesquisasFuturas}

\section{Introdu��o}


\section{Tomada de decis�o}
\label{sec:tomadaDecisao}


	\bibliographystyle{apalike}	
	\bibliography{Bibliografia}

\end{document}