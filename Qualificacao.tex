\documentclass{ufabc}

\usepackage{amscd}
\usepackage{amsfonts}
\usepackage{amssymb}
\usepackage{epsf}
\usepackage{graphicx}
\usepackage{algorithm}
\usepackage{algpseudocode}
\usepackage{algpascal}
\usepackage{indentfirst}
\usepackage{amsmath}
\usepackage[only,ninrm,elvrm,twlrm,fivrm,sixrm,sevrm,egtrm,egtit,tenrm,tenit,twlit,frtnrm]{rawfonts}

\instituicao{Universidade Federal do ABC\\Centro de Matem�tica, Computa��o e Cogni��o (CMCC)}
\curso{P�s-Gradua��o}{Ci�ncia da Computa��o}{Mestre}
\titulo{Agentes inteligentes para batalhas Pok�mon}
\autor{Jonathan Ohara}{de Araujo}
\orientador{Orientador: Prof. Dr. Fabr�cio Olivetti de Fran�a}
\coordenador{Prof. Dr. Jo�o Paulo G�is}	
\documento{Qualifica��o}
\convidados{Prof. Dr. Fabr�cio Olivetti de Fran�a}{Prof.Dr. Jo�o Paulo Gois}{Prof.Dr. Andr� Luiz Brand�o}{Prof.Dr. Denise Hideko Goya (Suplente)}
	
\date{03}{2016}

% novos comandos para qualificacao
\newcommand{\kCC}{\emph{kCC}}
\newcommand{\kCH} {\emph{kCH}}
\newcommand{\uCC}{\emph{1CC}}
\newcommand{\uCH} {\emph{1CH}}
\newcommand{\dCC}{\emph{2CC}}
\newcommand{\dCH} {\emph{2CH}}
\newcommand{\PTH}{\mathcal{P}}
\newcommand{\Tau}{\mathcal{T}}
\newcommand{\NP}{{\rm NP}}

\begin{document}
	\maketitle

	\pagenumbering{roman}
	
	\begin{resumo}
Esse trabalho explora a cria��o de agentes inteligentes competitivos em um ambiente onde dezenas de milhares de jogadores humanos competem diariamente para melhorar sua coloca��o no sitema de ranqueamento do jogo Pok�mon Showdown.

Escolher o melhor algoritmo e a melhor forma de aprendizado para esse agente � um dos grandes desafios desse trabalho. Outro importante aspecto que o agente precisa se adaptar � a grande variedade de composi��es de times e Pok�mons que o agente e o seu advers�rio pode montar.

Para obter a melhor flexibilidade o agente ser� submetido somente a batalhas rand�micas onde as composi��es de equipes s�o todas aleat�rias e a avalia��o de performance do agente ser� feita atrav�s do sistema de ranqueamento do pr�prio jogo.
\end{resumo}
	\begin{resumoingles}
This work explores the creation of an API (Application Programming Interface) to the online strategy game called Pok\'{e}mon Showdown!, allowing researchers to create autonomous agents to compete against human players in an extremely competitive environment with thousands of active players at all the time. The creation of this API can enable searches of new artificial intelligence techniques and comparision of different techniques when submitted to games with humans.

In this work we discuss and develop agents for this game using different Monte Carlo search tree techniques. There some important aspects need to be taken care during the creation of agents for this game such as: incompleteness and imperfection of information and simultaneity of choices are some of the main challenges.
\end{resumoingles}

	\tableofcontents
	
	\listoffigures
	
	\listoftables

	\newpage
	
	\pagenumbering{arabic}

	\chapter{Introducao}

Essa � uma intro
	\chapter{Agentes Inteligentes}
\label{cap:agentesInteligentes}

Na computa��o, especialmente em intelig�ncia artificial, � bem comum utilizar-se do termo agente ou agente inteligente. Por defini��o do dicion�rio \cite{michaelis} agente significa "Que age, que exerce alguma a��o; que produz algum efeito".

Na academia temos defini��es mais direcionadas como a de \cite{aiModenApproach} "Um agente � algo capaz de perceber seu ambiente atrav�s de sensores e agir sobre esse ambiente por meio de atuadores", ou a de \cite{kidsim} "Vamos definir um agente como uma entidade de \textit{software} persistente dedicada para um prop�sito espec�fico. 'Persistente' distingue agentes de sub-rotinas; agentes t�m seus pr�prias ideias sobre como realizar tarefas, suas pr�prias agendas. 'Prop�sitos especiais' os distingue de aplica��es multifuncionais inteiras; agentes s�o tipicamente muito menores."

Existem diversas defini��es de agentes. Mas qual seria a diferen�a de um agente para um programa de computador? Os autores \cite{agentOrProgram} prop�em uma defini��o mais s�lida para agente "Um agente aut�nomo � um sistema situado dentro e como parte de um ambiente que sente esse ambiente e age sobre ele e, ao longo do tempo, busca de sua pr�pria agenda de modo a efetivar o que sente no futuro."  ainda segundo os autores maioria dos programas comuns violam uma ou mais regras dessa defini��o, por exemplo um sistema de folha de pagamento sente o ambiente atrav�s de suas entradas e age sobre ele gerando uma sa�da, mas n�o � um agente porque sua sa�da normalmente n�o afeta o que ele sente mais tarde.

\section{Classifica��es de agentes}

Os agentes possuem tamb�m diferentes explica��es quanto a suas caracter�sticas, classifica��es e taxonomia. Uma das defini��es mais antigas e mais s�lidas � a de \cite{Nwana96softwareagents} nesse trabalho o autor define tr�s dimens�es para o agente:

\begin{itemize}

	\item \textbf{Mobilidade}. Capacidade de se mover em torno de alguma rede. Definindo agentes como fixos ou m�veis;
	
	\item \textbf{Deliberativos ou reativos}. Agentes deliberativos possuem s�mbolos  internos, modelos de racioc�nio e se envolvem com o planejamento e negocia��o a fim de conseguir uma coordena��o com outros agentes. J� os reativos agem por est�mulos/respostas que o estado atual do ambiente que est� inserido o proporciona.
	
	\item \textbf{Ideais e atributos prim�rios}. Os agentes podem ser classificados de acordo com tr�s caracter�sticas: autonomia, cooperatividade e aprendizagem. Autonomia se refere ao princ�pio que o agente deve realizar a��es sem interven��o humana. Coopera��o � a habilidade de se comunicar com outros agentes. Aprendizagem � a capacidade de aprender conforme informa��es s�o lidas e a��es sejam realizadas. Dessas defini��es surgem quatro tipos de agentes conforme mostra a Figura \ref{fig:tiposagentes}.
	
\end{itemize}

\begin{figure}[!h]
\centering
\includegraphics[width=16cm]{figures/tiposagentes.png}
\caption[Vis�o parcial da tipologia de um Agente]{Vis�o parcial da tipologia de um Agente.}
\label{fig:tiposagentes}
\end{figure}

J� \cite{agentOrProgram} classifica os agentes pelas propriedades que eles apresentam, s�o elas: reatividade, autonomia, orienta��o a objetivo, tempor�ria ou cont�nua, comunicativa, aprendizagem, mobilidade, flexibilidade e car�ter.

Embora as diferentes defini��es, muitas coisas s�o comuns ou muito parecidas como a presen�a de caracter�sticas como autonomia, aprendizado, objetivos e comunicabilidade. 

\section{Agentes em jogos}

Antigamente maioria do processamento do sistema que rodava um jogo (computador, \textit{videogame} entre outros) era utilizada para processamentos gr�ficos. Com a evolu��o e custo baixo dos equipamentos (\textit{hardware}) hoje em dia � poss�vel dar mais aten��o a outros aspectos de um jogo. O autor \cite{iaInGames} discorre um outro ponto "Como jogos de computador se tornam mais complexos e os consumidores exigem advers�rios mais sofisticados controlados pelo computador, os desenvolvedores de jogos s�o obrigados a colocar uma maior �nfase nos aspectos de intelig�ncia artificial de seus jogos". Por isso hoje, existem um grande n�mero de jogos que n�o s�o conhecidos por sua beleza gr�fica, segundo \cite{aiForGames} um n�mero cada vez maior de jogos tem como principal ponto do jogo a Intelig�ncia Artificial, como por exemplo Creatures[Cyberlife Technology Ltd., 1997] em 1997, e jogos como The Sims[Maxis Software, Inc., 2000] e Black and White [Lionhead Studios Ltd., 2001]. Ainda segundo \cite{aiForGames} o jogo Creatures foi um dos sistemas mais complexos de intelig�ncia artificial visto em um jogo, com um c�rebro baseado em rede neural para cada criatura presente no jogo.

Agentes para jogos tem uma infinidade de prop�sitos. Ao associar esses dois termos � bem comum pensar em oponentes controlados por agentes, por�m a utiliza��o de agentes vai muito al�m. � poss�vel utiliza��o de agentes em outros aspectos como na gera��o de conte�do procedural, cria��o de desafios e miss�es entre outros.

O artigo \textit{Procedural Content Generation for Games: A survey} \cite{PCGsurvey2013} cita que a gera��o de conte�do atrav�s de agentes inteligentes permite a cria��o de diferentes tipos de terrenos sem perder o controle do \textit{design} do jogo.

O trabalho de \cite{questGenerator2011} explora a gera��o procedural de miss�es e desafio em jogos(especialmente para jogos de MMORPGs \textit{online}), segundo os autores essa abordagem tem o potencial de aumentar a variedade e a longevidade de um jogo.

Existem tamb�m outras aplica��es para agentes como busca de caminhos em ambientes complexos, gera��o de objetos, danifica��o de objetos entre outras.

\section{Agentes inteligentes contra jogadores}
\label{sec:agentesInteligentescontrajogadores}

Existem v�rias abordagens para cria��o de agentes que enfrentam jogadores. Podem-se enumerar alguns deles como agentes adaptativos, evolutivos e competitivos.

Em alguns jogos o jogador n�o tem uma op��o de dificuldade, ficando a crit�rio do sistema do jogo controlar a dificuldade, como nos jogos \textit{Middle Earth: Shadow of Mordor} e \textit{Max Payne 2}. Segundo \cite{Ponsen04improvingadaptive} no jogo Max Payne 2 � introduzido algo chamado op��es din�micas de dificuldade. Informa��es do mundo do jogo s�o extra�das para estimar a habilidade do jogador, ajustando a dificuldade da intelig�ncia artificial.

Alguns agentes s�o criados para acompanhar a curva de aprendizado de cada pessoa, para que o jogador sempre tenha um bom n�vel de desafio sempre que jogar. No jogo \textit{Star Craft 2} existe o modo chamado pareamento contra Intelig�ncia Artificial, nesse modo o desempenho do jogador em partidas passadas define qual o n�vel do jogador, ou seja, quanto mais o jogador evolui mais dif�cil ser� seu oponente.

Outra abordagem interessante � a utilizada em Drivatar \cite{microsoft:drivatar}, inserida no jogo Forza Motorsport, � um modulo que cria agentes "humanizados"  que aprendem com a pessoa que est� jogando, incorporando suas caracter�sticas e utilizando para jogar contra o pr�prio jogador. 

Este trabalho explora a cria��o de agentes competitivos, ou seja, agentes projetados para ganhar de jogadores humanos. Segundo \cite{gamesComputerAI1} apenas na d�cada de 90 os computadores foram capazes de competir com sucesso contra o melhor dos humanos. O primeiro jogo a ter um campe�o mundial n�o humano foi o jogo de damas em 1994 seguido pelo Deep Blue em 1997.

O agente jogador mais conhecido � o Deep Blue da IBM. O artigo Deep Blue: System overview \cite{deepblue1} explica o funcionamento do agente. O Deep Blue � composto por um \textit{chip} chamado \textit{Chess Chip} que � dividido em tr�s partes:

\begin{itemize}

	\item \textbf{Buscador alphabeta}. Menor parte do \textit{chip}, contendo apenas 5\% de seu total, esse componente � respons�vel por fazer a busca na �rvore de jogadas. O algoritmo utilizado � uma variante do \textit{alphabeta} chamado de \textit{minimum-window alpha beta search}, essa t�cnica foi escolhida por n�o ser necess�rio uma pilha de valores.
	
	\item \textbf{Gerador de movimentos}. Esse componente cont�m o \textit{array} bidimensional 8x8 referente a cada posi��o do tabuleiro do xadrez, ele tamb�m � respons�vel por realizar e verificar a legalidade dos movimentos de acordo com as regras do xadrez.
	
	\item \textbf{Fun��o de avalia��o}. Ocupando cerca de dois ter�os do \textit{chip} esse componente � respons�vel pela avalia��o dos movimentos. Cada poss�vel posi��o de cada pe�a � avaliado, essa avalia��o � baseada em quatro crit�rios: material, posi��o, seguran�a do rei e tempo \cite{ibmdeepblue}. Material � o quanto cada pe�a "vale". Por exemplo, um pe�o vale 1 enquanto uma torre vale 5. Posi��o quantifica o qu�o bom est�o posicionadas as pe�as, nesse c�lculo uma s�rie de aspectos � levada em conta, um deles � o n�mero de movimentos seguros que as pe�as podem fazer para atacar. Seguran�a do rei � calculo que leva em conta a prote��o do rei. Por final tempo, al�m de controlar o tempo da pr�pria jogada, fazer uma jogada r�pida d� ao advers�rio menos tempo para pensar em poss�veis jogadas.
	
\end{itemize}

Como pode ser visto no agente do Deep Blue a heur�stica � algo muito importante na constru��o de um agente para jogos complexos. Em alguns jogos a avalia��o de heur�stica n�o � t�o importante, um exemplo desse tipo de jogo � o jogo da velha, nele temos uma �rvore pequena de possibilidades e o resultado � vit�ria, empate ou derrota, ou seja, n�o � necess�rio que a cada n� da �rvore seja feita uma complexa avalia��o de heur�stica.

Segundo \cite{Nielsen:1994:HE:189200.189209} avalia��o heur�stica envolve ter um pequeno conjunto de avaliadores examinando a interface e avaliando a sua conformidade em rela��o ao resultado. 

Os autores \cite{DBLP:conf/aaai/ChristensenK86} discorrem que, em jogos de tabuleiro de duas pessoas a fun��o de heur�stica � vagamente caracterizado pela "for�a" do posicionamento de um jogador contra o outro. Ainda segundo \cite{DBLP:conf/aaai/ChristensenK86} pode-se dizer que uma fun��o de avalia��o de heur�stica tem duas propriedades:

\begin{itemize}
	\item Quando aplicado a um estado final (no caso de uma �rvore, um n� folha), a avalia��o de heur�stica tem que devolver o estado corretamente;
	\item O valor da fun��o de heur�stica � invari�vel ao longo de um caminho de solu��o �tima.
\end{itemize}

\section{Competi��o de agentes inteligentes}

Um recurso bastante utilizado para pesquisa em intelig�ncia artificial � o desenvolvimento de agentes inteligentes que compitam com outros agentes que utilizem diferentes t�cnicas. Tais competi��es s�o bem comuns nos grandes congressos de intelig�ncia artificial. 

As competi��es de intelig�ncia para Start Craft que ocorrem desde 2010 avan�ou significativamente o campo de intelig�ncia artificial para jogos de estrat�gia em tempo real segundo \cite{6637024}. Em Star Craft os jogadores tem que se preocupar em construir um ex�rcito ao mesmo tempo em que gere uma economia bem complexa. Nesse jogo o vencedor � definido pelo jogador que derrotar todas as tropas e estruturas advers�rias. Nessa competi��o dois agentes s�o colocados para guerrear at� que haja um vencedor. A comunica��o com o jogo � feito pela BWAPI uma API em C++ que permite comunica��o do agente com o jogo, ou seja, atrav�s dessa biblioteca � poss�vel realizar comandos que ser�o executados no jogo.

A figura \ref{fig:starcraftai} mostra a API funcionando. Ele usa o sistema de conversa do jogo para mostrar mensagens do agente.

\begin{figure}[!h]
\centering
\includegraphics[width=16cm]{figures/bwapi.png}
\caption[BWAPI - StarCraft API]{BWAPI em funcionamento.}
\label{fig:starcraftai}
\end{figure}

Outra competi��o um pouco diferente � a AI Birds \cite{aibirds1}, nessa competi��o os agentes s�o submetidos a jogar Angry Birds, o objetivo desse jogo � voc� destruir objetos(ganhando assim pontos) com p�ssaros que s�o arremessados por um estilingue. Diferente do Star Craft nessa competi��o n�o h� um enfretamento direto entre os agentes, eles s�o submetidos a enfrentar o mesmo cen�rio e vence o agente que obtiver mais pontos. A comunica��o com o jogo � bastante diferente do Star Craft, nesse caso o jogo roda via navegador de internet (apenas Google Chrome) e um \textit{plugin} de JavaScript tira fotos do jogo e passa essa imagem via WebSocket para uma linguagem que ir� interpretar tal foto.

Na figura \ref{fig:angrybirdsai} pode ser visto como a API se comunica com jogo. Uma imagem � capturada do navegador e a partir disso � utilizado um algoritmo de reconhecimento de imagem para identificar o que � cada objeto (os quadrados em volta do objeto s�o padr�es que a API est� reconhecendo).

\begin{figure}[!h]
\centering
\includegraphics[width=16cm]{figures/aibirds.png}
\caption[AIBIRDS - Angry Birds AI API]{Api Angry Birds AI em funcionamento.}
\label{fig:angrybirdsai}
\end{figure}

%ID�IA: Talvez comentar de outra competi��o como Geometry Friends

Neste trabalho al�m da cria��o de agentes, ser� explorada a cria��o de uma API para comunica��o com o jogo Pok�mon Showdown, um simulador de batalhas Pok�mon. No cap�tulo \ref{cap:batalhasPokemon} ser� explicado como funciona o sistema de batalhas. A API possibilita batalhas contra outros agentes e contra jogadores humanos. A API est� escrita em JavaScript, mas tamb�m haver� possiblidade de comunica��o via WebSocket, onde qualquer linguagem que tenha o recurso poder� fazer um agente.
	\chapter{Intelig�ncia Artificial}
\label{cap:inteligenciaArtificial}

Neste cap�tulo � feito uma revis�o das t�cinicas de intelig�ncia artificial que ser�o utilizadas nesse trabalho. Algoritmos baseados em grafos \ref{sec:algoritmoGrafos}, aprendizado por refor�o \ref{sec:aprendizadoReforco} e neuroevolu��o \ref{sec:neuroevolucao}.

\section{Intelig�ncia para jogos}
\label{sec:inteligenciaJogos}

O estudo de IA/IC (Intelig�ncia artificial/Intelig�ncia computacional) para jogos � uma �rea de estudo que vem crescendo muito na �ltima d�cada. Grandes confer�ncias como o IEEE Conference on Computational Intelligence and Games (CIG), AAAI Aritificial Intelligence and Interactive Digital Entertainment (AIIDE) e IEEE Transactions on Computational Intelligence and AI in Games (TCAIG) j� contam com mais de dez edi��es.

Jogos podem ser usados como cen�rio desafiador para avalia��o de m�todos de intelig�ncia computacional, pois eles prov�m elementos din�micos e competitivos que s�o pertinentes ao mundo real \cite{cig2014}.

Segundo \cite{panoramaAIGames} durante so semin�rios em Dagstuhl(centro de pesquisa de ci�ncia da computa��o alemanha) foi poss�vel identificar dez grandes t�picos em IA/AC:

\begin{itemize}
	\item  Aprendizado de comportamento para jogadores n�o humanos (JNH); 
	\item  Busca e planejamento;
	\item  Modelagem de personagens;
	\item  Jogos como avalia��o comparativa para Intelig�ncia Artificial (IA);
	\item  Gera��o de conte�do procedural;
	\item  Narrativa computacional;
	\item  Agentes cr�veis;
	\item  \textit{Game design} assistido por IA;
	\item  IA para jogos em geral;
	\item  IA em jogos comerciais.
\end{itemize}

Ainda segundo \cite{panoramaAIGames} todas as �reas de pesquisa podem ser vistas como potenciais influenciadores delas mesmas em algum grau. Essas conex�es e interconex�es geram outras �reas de pesquisa.

Nas pr�ximas se��es ser� feito uma revis�o dos principais m�todos que � utilizado nos jogos e que ser�o usados neste trabalho.

\section{Algoritmos baseados em grafos}
\label{sec:algoritmoGrafos}

Um dos grandes desafios de um  agente e de um jogador dentro de um jogo � a tomada de decis�o, expandindo um pouco essa ideia, pode-se dividir a tomada de decis�o em partes menores: levantar op��es, avaliar as op��es e escolher qual pode conduzir o jogador a vit�ria dado um determinado cen�rio.

Uma algoritmo muito comum para esse cen�rio � o \textit{minimax}. Segundo \cite{browne12asurvey} jogos do mundo real normalmente envolvem uma estrutura de recompensas em que apenas as recompensas obtidas em estados terminais(jogadas que definem quem � o vencedor) do jogo descrevem com precis�o o qu�o bem cada jogador est� se saindo. Os jogos s�o, portanto, normalmente modelado como �rvores de decis�es da seguinte forma:

\begin{itemize}
	
	\item \textbf{Minimax} tenta minimizar recompensa m�xima do oponente em cada estado, e � a abordagem tradicional para pesquisa de jogos combinat�rios em dois jogadores.
	
	\item \textbf{Expectimax} generaliza \textit{minimax} para jogos estoc�sticos em que as transi��es de estado para estado s�o probabil�stica. O valor de um n� � a soma dos valores dos n�s filhos ponderados por suas probabilidades(possibilidade de um estado ocorrer). Estrat�gias de poda de �rvore mais complexas devido a probabildade de um n� acontecer.

	\item \textbf{Miximax} � semelhante ao \textit{expectimax} de apenas um jogador e � usado principalmente em jogos com informa��es n�o precisas. Ele usa uma estrat�gia  predefinida para tratar a decis�o do oponente como n�s probabil�sticos.
	
\end{itemize}

%Falar Sobre como funciona o Minimax

%Falar sobre evolucoes do minimax (alphabeta)

%falar do Monte Carlo Tree Search

\section{Aprendizado por refor�o}
\label{sec:aprendizadoReforco}

\section{Neuroevolu��o}
\label{sec:neuroevolucao}
	\chapter{Batalhas Pok�mon}
\label{cap:batalhasPokemon}

	\chapter{Metodologia}
\label{cap:proposta}

Para atingir os objetivos propostos ser�o implementados tr�s agentes. Dois desses ser�o agentes que ir�o ser submetido a uma grande quantidade de treino para que possam chegar a patamares s�lidos. O terceiro ser� um agente baseado em grafos com diferentes profundidades onde pode ser visto a evolu��o no ranquemanto com o aumento de profundidade de �rvore de decis�o.

\begin{itemize}

	\item \textbf{Agente 1 com neuroevolu��o}. O primeiro agente persiste em um rede neural onde os pesos de sua rede ser�o ajustados por um algoritmo evolutivo.
	
	\item \textbf{Agente 2 com aprendizado por refor�o}. Esse agente ir� se ajustar atrav�s de estimulos positivos e negativos que regular�o sua decis�o dentro do jogo.
	
	\item \textbf{Agente 3 baseado em grafo de decis�o}. Utilizar� uma �rvore de decis�o onde os poss�veis movimentos ser�o mapeados. O algoritmo tentar� prever poss�veis inc�gnitas do advers�rio assumindo o pior cen�rio poss�vel para cada vari�vel n�o conhecida..
	
\end{itemize}

\section{Treino e aprendizado}

O primeiro agente que ser� testado contra jogadores humanos � o agente 3(grafo). Ser� criado v�rias vers�es desse agente, cada vers�o ser� cablibrado com uma profundidade diferente em sua �rvore. Cada vers�o do agente ser� submetido a uma s�rie da batalhas que s� ter� fim quando seu ranqueamento se estabilizar.

Os agentes 1 e 2 ter�o comportamentos semelhantes em como ser�o treinados. Ambos agentes ter�o duas vers�es, cada vers�o far� um treinamento diferente:

\begin{itemize}
	\item \textbf{Treino contra humanos}: O agente ser� submetido a jogar contra jogadores humanos que ser� escolhido pelo pr�prio jogo \textit{Pok�mon Showdown!}, o sistema de pareamento de partidas � feito pelo ranqueamento dos dois jogadores, ou seja, o advers�rio sempre vai ter uma pontua��o no \textit{rank} pr�xima da do agente.
	\item \textbf{Treino contra agente 3}: Os agentes 1 e 2 jogar�o apenas contra o agente 3, sempre que o agente conseguir um grande n�meros de vit�ria sobre o agente 3 esse agente ser� modificado adicionando mais um na profundidade de sua �rvore.
\end{itemize}

\section{Avalia��o de resultado}

Avaliar a situa��o da batalha ser� um recurso muito importante pois o agente 3 precisar� avaliar cada n� de sua �rvore para fazer a melhor escolha ou poss�vel ou a escolha que gere menos boas escolhas para o advers�rio.Essa avalia��o ter� como base a quantidade de pok�mons vivos e a situa��o dele(quantidade de pontos de vida, afetado por alguem efeito negativo entre outros).

Ao fim de cada batalha tamb�m feita uma avalia��o da vit�ria. Com essa avalia��o ser� poss�vel aferir o qu�o dispar foi a situa��o do vencedora. Al�m disso, com essa m�trica podemos eliminar poss�veis ru�dos em resultados de batalhas como desist�ncias ou desconex�es por parte dos adver�rios.
%falar sobre quantificar a vitoria
	\chapter{Plano de Trabalho}

Essa � uma Planos

	\bibliographystyle{apalike}	
	\bibliography{Bibliografia}

\end{document}