\documentclass{ufabc}

\usepackage{amscd}
\usepackage{amsfonts}
\usepackage{amssymb}
\usepackage{epsf}
\usepackage{graphics}
\usepackage{algorithm}
\usepackage{algpseudocode}
\usepackage{algpascal}
\usepackage[only,ninrm,elvrm,twlrm,fivrm,sixrm,sevrm,egtrm,egtit,tenrm,tenit,twlit,frtnrm]{rawfonts}

\instituicao{Universidade Federal do ABC\\Centro de Matem�tica, Computa��o e Cogni��o (CMCC)}
\curso{P�s-Gradua��o}{Ci�ncia da Computa��o}{Mestre}
\titulo{Agentes inteligentes para batalhas Pok�mon}
\autor{Jonathan Ohara}{de Araujo}
\orientador{Orientador: Prof. Dr. Fabr�cio Olivetti de Fran�a}
\coordenador{Prof. Dr. Jo�o Paulo G�is}
\documento{Qualifica��o}
\convidados{Prof. Dr. Fabr�cio Olivetti de Fran�a}{Prof.Dr.B}{Prof.Dr.C}{Prof.Dr.D}

\date{03}{2016}

% novos comandos para qualificacao
\newcommand{\kCC}{\emph{kCC}}
\newcommand{\kCH} {\emph{kCH}}
\newcommand{\uCC}{\emph{1CC}}
\newcommand{\uCH} {\emph{1CH}}
\newcommand{\dCC}{\emph{2CC}}
\newcommand{\dCH} {\emph{2CH}}
\newcommand{\PTH}{\mathcal{P}}
\newcommand{\Tau}{\mathcal{T}}
\newcommand{\NP}{{\rm NP}}

\begin{document}
	\maketitle

	\pagenumbering{roman}
	
	\begin{resumo}
Esse trabalho explora a cria��o de agentes inteligentes competitivos em um ambiente onde dezenas de milhares de jogadores humanos competem diariamente para melhorar sua coloca��o no sistema de ranqueamento do jogo Pok�mon Showdown.

Escolher o melhor algoritmo e a melhor forma de aprendizado para jogar contra humanos � um dos grandes desafios desse trabalho. Outro importante aspecto que o agente precisa se adaptar � a grande variedade de composi��es de times e Pok�mons que o agente e o seu advers�rio pode montar.

Para obter a melhor flexibilidade o agente ser� submetido somente a batalhas rand�micas onde as composi��o das equipes s�o todas aleat�rias, e a avalia��o de performance do agente ser� feita atrav�s do sistema de ranqueamento do pr�prio jogo.
\end{resumo}
	\begin{resumoingles}

To Do.

\end{resumoingles}

	\tableofcontents
	
	\listoffigures
	
	\listoftables

	\newpage
	
	\pagenumbering{arabic}

	\chapter{Introdu��o}
\label{cap:introducao}

\section{Motiva��o}
\label{sec:motivacao}

Existem diversas finalidades para cria��o de agentes inteligentes para jogos. Podemos enumerar algumas como: adapta��o a diferentes tipos de jogador, gerar diferentes experi�ncias em cada nova partida, gerar grandes desafios entre outras.

Esse trabalho explora a cria��o de agentes inteligentes que compitam com jogadores humanos no sistema de batalhas \textit{Pok�mon} no ambiente \textit{Pok�mon Showdown}.

Um dos grandes desafios na cria��o desses agentes � a adaptabilidade e a competitividade. Diferentes de jogos como xadrez, damas, reversi e outros jogos de tabuleiro, em batalhas \textit{Pok�mon} nada se sabe do advers�rio at� que comece o jogo, ou seja, invalidando qualquer tipo de t�cnica prevendo poss�veis movimentos antes do jogo come�ar j� que cada time pode ser montado de uma infinidade de modos diferentes. Para acentuar ainda mais essas proprieaddes os agentes ir�o treinar e competir no modo rand�mico, nesse modo a escolha do seu time assim como de seu advers�rio � feito pelo pr�prio sistema do jogo.

Por causa da caracter�stica de desconhecimento do time advers�rio, a utiliza��o de t�cnicas de cria��o de �rvores de poss�veis jogadas do advers�rio � bastante prejudicada, pois o agente precisaria predizer os poss�veis \textit{Pok�mons} advers�rios assim como suas caracter�sticas, assim dificultando a utiliza��o da t�cnica que ficou famosa pelo sistema \textit{Deep Blue} que segundo o trabalho \textit{Deep Blue System Overview} \cite{deepblue1} "O \textit{Deep Blue} � um massivo sistema paralelo para realiza��o de busca em �rvores de jogos de xadrez".

\section{Objetivos}
\label{sec:objetivos}
O Objetivo desse trabalho � o desenvolvimento de agentes inteligentes que joguem e aprendam com milhares de jogadores humanos. Ser� implementada distintas t�cnicas para cria��o e aprendizado desses agentes. No decorrer do trabalho ser� sumarizado a evolu��o dos agentes no sistema de ranqueamento do jogo e, essa posi��o ser� confrontada com a quantidade de treinamento que cada agente recebeu, podendo assim observar a curva de melhora em rela��o a quantidade de treinamento.

Para cria��o desses agentes foi desenvolvida uma API (Application Programming Interface) que permitir� a comunica��o com o jogo \textit{Pok�mon Showdown}. Inicialmente a API est� disponibilizada apenas para JavaScript mas durante o desenvolvimento do projeto ser� portada para Java atrav�s de WebSockets.

\section{Principais contribui��es}
\label{sec:principais}

O trabalho ir� explorar a cria��o de agentes inteligentes adaptativos, num ambiente que pouco se sabe sobre o advers�rio. Os agentes ter�o informa��es apenas durante a batalha e, essas informa��es s�o apenas aquelas que o advers�rio realizar. Por exemplo: o agente s� saber� que advers�rio tem um \textit{Pok�mon} at� o mesmo us�-lo, os movimentos que \textit{Pok�mon} tem ser�o apenas conhecidos a medida que o advers�rio utiliz�-los e existem caract�risticas que agente n�o tem como descobrir como por exemplo a quantidade de ataque e defesa distruibida no monstro.

Al�m disso a constru��o de uma API para acesso ao jogo contrubuir� para que outros pesquisadores tamb�m possam desenvolver estudos e criar seus pr�prios agentes podendo criar-se uma cultura de competi��o entre agente inteligentes na plataforma \textit{Pok�mon Showdown}.


	\chapter{Contextualizacao}

Essa � uma Contextualizacao
	\chapter{Revis�o Bibliogr�fica}

Rev. Bibliogr�fica: leitura e resumo de todos os trabalhos relacionados. Nele vc vai descrever artigos que aplicam t�cnicas de IA em joos do mesmo estilo, artigos que tentam aplicar para esse mesmo jogo (talvez n�o tenha), artigos de t�cnicas que ser�o utilizadas nesse seu trabalho.
%	\chapter{Objetivos}

Essa � uma Objetivos
	\chapter{Metodologia}
\label{cap:metodologia}

Para atingir os objetivos propostos ser�o implementados tr�s agentes. Dois desses agentes ser�o submetidos a uma grande quantidade de partidas de treinamento para que possam chegar a patamares est�veis. O terceiro ser� um agente baseado em grafos com diferentes profundidades, onde ser� categorizada a evolu��o no ranquemanto do jogo de acordo com o aumento de profundidade da �rvore de decis�o.

\begin{itemize}

	\item \textbf{Agente 1 com neuroevolu��o}. O primeiro agente persiste em uma rede neural onde os pesos de sua rede ser�o ajustados por um algoritmo evolutivo.
	
	\item \textbf{Agente 2 com aprendizado por refor�o}. Esse agente ir� se ajustar atrav�s de est�mulos positivos e negativos que regular�o sua decis�o dentro do jogo atrav�s do algoritmo de aprendizado por refor�o.
	
	\item \textbf{Agente 3 baseado em �rvore de decis�o}. Utilizar� uma �rvore de decis�o (algoritmo baseado em \textit{minimax}) onde os poss�veis movimentos ser�o mapeados. O algoritmo tentar� prever poss�veis inc�gnitas do advers�rio assumindo o pior cen�rio poss�vel para cada vari�vel n�o conhecida.
	
\end{itemize}

\section{Treino e aprendizado}
\label{sec:treino}

O primeiro agente que ser� testado contra jogadores humanos � o agente 3 (�rvore). Ser�o criadas diferentes vers�es desse agente. Cada diferente vers�o analisar� diferentes profundidades de �rvores, ou seja, ser�o criados $N$ vers�es do agente, a primeira vers�o analisar� jogadas at� a $x$ profundidade da �rvore, a pr�xima vers�o analisar� at� a $x + 1$ profunidade. O jogo tem um limite m�ximo de dois minutos para fazer sua jogada, com base nesse limite de tempo ser�o reguladas quantas diferentes vers�es dessa t�cnica ser�o criados.

Para executar essa abordagemser� criado uma fun��o de avalia��o de heur�stica baseado nas informa��es da se��o \ref{sec:batalhaspokemon}. Essa t�cnica ser� importante para evitar problema de grande tempo de execu��o. Al�m disso, poderemos confrontar os outros dois agentes com diferentes vers�es desse agente para comparar o n�vel de aprendizado de cada agente.

Os agentes 1 e 2 ter�o comportamentos semelhantes em como ser�o treinados. Ambos agentes ter�o duas vers�es, cada vers�o ser� submetida aos seguintes treinamentos:

\begin{itemize}
	
	\item \textbf{Treino contra humanos}: O agente ser� submetido a jogar contra jogadores humanos que ser�o escolhidos pelo pr�prio jogo, o sistema de pareamento de partidas � feito baseado no ranqueamento dos dois jogadores, ou seja, o advers�rio sempre vai ter uma pontua��o no \textit{rank} semelhante ao agente.
	
	\item \textbf{Treino contra agente 3}: Os agentes 1 e 2 jogar�o contra o agente 3, sempre que o agente que est� treinando conseguir um grande n�meros de percentagem de vit�ria sobre o agente 3, ser� aumentado mais um n�vel de profundidade na �rvore do agente 3.

\end{itemize}

\section{Avalia��o de resultado}

Avaliar a situa��o da batalha ser� um recurso muito importante, pois o agente 3 precisar� avaliar cada n� de sua �rvore para fazer a melhor escolha poss�vel, ou a escolha que gere menos boas escolhas para o advers�rio. Essa avalia��o ter� como base a quantidade de Pok�mons vivos e a situa��o deles (quantidade de pontos de vida, afetado por algum efeito negativo entre outros).

Ao fim de cada batalha tamb�m ser� feita uma avalia��o do jogo. Com essa avalia��o ser� poss�vel aferir o qu�o d�spar foi � situa��o do vencedor, al�m disso, com essa m�trica podemos eliminar poss�veis ru�dos em resultados de batalhas como desist�ncias ou desconex�es por parte dos oponentes.
	\chapter{Plano de Trabalho}

Plano de trabalho e cronograma: itemize e coloque no diagrama de cronograma tudo que vc j� fez relacionado a essa pesquisa (ex.: cite o estudo de caso do angry birds, a elabora��o da interface com o jogo) e os pr�ximos passos. Coloque nesse cronograma a previs�o de elabora��o de um artigo e da disserta��o.Plano de trabalho e cronograma: itemize e coloque no diagrama de cronograma tudo que vc j� fez relacionado a essa pesquisa (ex.: cite o estudo de caso do angry birds, a elabora��o da interface com o jogo) e os pr�ximos passos. Coloque nesse cronograma a previs�o de elabora��o de um artigo e da disserta��o.
	\chapter{Cronograma}

Essa � uma Cronograma

	\bibliographystyle{apalike}	
	\bibliography{Bibliografia}

\end{document}